% ================================ Lecture 4 ===================================
\flashcard{\normalfont
  \textbf{Conventional/ symmetric} cryptography is based on \blank{\textbf{shared secret keys}}.
}

\card{\normalfont
  What are the problems with \textbf{symmetric/ conventional cryptography}?
}{\normalfont
  \begin{flushleft}
    A separate key is needed for each pair of users (or even for each ciphertext encryption -> session key) \\
    - So an n-user system requires n*(n-1)/2 keys -> the n2 problem. \\
    - Generating and distributing these keys are a challenging problem. \\
    - Maintaining security for the keys already distributed is also challenging: can one remember so many keys? \\
    As two parties share the same key, non-repudiation can not be achieved.
  \end{flushleft}
  Problems with symmetric/ conventional cryptography
}

\flashcard{\normalfont
  \blank{\textbf{Public key}} cryptosystems are based on mathematically hard problems rather than substitution/ transportation (permutation) ciphers.
}

\card{\normalfont
  How does \textbf{public-key} cryptography work?
}{\normalfont
  \begin{flushleft}
    A pair of keys is used: One is private (secret), the other can be made public. The pair of private and public keys are related mathematically. It is infeasible to generate one from the other. \\
    Encryption generated with one key can only be decrypted with the other key in the pair.
  \end{flushleft}
  Public-key cryptography
}

\flashcard{\normalfont
  \blank{\textbf{Confidentiality}} protection is only applied to short messages, e.g. for secure transportation of symmetrical keys. \\
  It is achieved by encrypting the plaintext M using the recipient's public key. Only the recipient has the corresponding private key so M can only be read by the recipient.
}

\flashcard{\normalfont
  A \blank{\textbf{signature}} is put on the hash value of M, and a \blank{\textbf{timestamp}} should also be included. This is used for message authentication and integrity and non repudiation of message origin. \\
  It it achieved by \blank{signing} M (actually the hash of it) using the sender's \textbf{private key}. Only the sender has the private key, so the message must have been signed by the sender.
}

\flashcard{\normalfont
  \blank{\textbf{Public key cryptography}} is based on the idea of a \textit{trapdoor} function, or mathematically hard problems.
}

\card{\normalfont
  Why is PKC using mathematically hard problems?
}{\normalfont
  \begin{flushleft}
    - Easy to generate keys (public and private) \\
    - Easy to encrypt and decrypt if the right key is known. \\
    - Hard to compute private key from public key. \\
    - Hard to recover plaintext from ciphertext without the right key.
  \end{flushleft}
  Reasons for using mathematically hard problems for PKC
}

\card{\normalfont
  What is the mathematical definition of \textbf{modulo}?
}{\normalfont
  \textit{a = b mod n}, which means there exists an integer number k such that a can be represented as a = k * n + b with the condition that 0 <= b <= n-1. \\
  Given integers a, b, and n != 0, a is congruent to b modulo n, if and only if a - b = k * n for some integer k, i.e. n divided (a-b). The notation is a = b (mod n). We call n the \textbf{modulus}, and b is \textbf{remainder} or \textbf{residue} of a modulo n.
}

\card{\normalfont
  What does the \textbf{RSA algorithm} consist of?
}{\normalfont
  \begin{flushleft}
    The algorithm consists of two numbers, the modulus (represented by the letter n) and the public exponent (represented by the letter e). \\
    The modulus is the product of two very large prime numbers (100 to 400 digits), represented by letters p and q (they need to be kept secret). \\
    It is a block cipher in which the plaintext and ciphertext are integers between 0 and n-1 for some n.
  \end{flushleft}
  RSA Algorithm
}

\card{\normalfont
  What are the three stepts of the \textbf{RSA algorithm}?
}{\normalfont
  \begin{flushleft}
    1. Key generation \\
    2. Encryption \\
    3. Decryption
  \end{flushleft}
  RSA Algorithm Steps
}

\card{\normalfont
  What is the pseudo-code for the \textbf{key generation} step of the \textbf{RSA algorithm}?
}{\normalfont \footnotesize
  \begin{flushleft}
    - Select two large primes (e.g. 200 digits) p and q \\
    - Calculate n = p * q and phi(n) = (p-1) * (q-1) \\
    - Select integer e relatively prime to phi(n) and 1 < e < phi(n) \\
    - Calculate d = e (to -1) mod phi(n) (or de = 1 mod phi(n)) \\
    - \textbf{Public key = (e, n)} \\
    - \textbf{Private key = (d, n)} \\
    - So: p and q are private and chosen; n= p * q is public and calculated (but keep p and q secret); e is public and chosen, and d is private and calculated.
  \end{flushleft}
  RSA Key Generation Step Pseudo Code
}

\card{\normalfont
  What is the pseudo-code for the \textbf{encryption} and \textbf{decryption} steps of the \textbf{RSA algorithm}?
}{\normalfont
  Encryption
  \begin{flushleft}
    - Represent the plaintext as an integer M in [0, n-1], i.e. M < n \\
    - Ciphertext: C = M (to e) mod n
  \end{flushleft}
  Decryption
  \begin{flushleft}
    - Ciphertext is C \\
    - Plaintext is M = C (to d) mod n
  \end{flushleft}
  RSA Encryption And Decryption Steps Pseudo Code
}

\card{\normalfont
  What does the \textbf{security} of the \textbf{RSA algorithm} rely on?
}{\normalfont \small
  \begin{flushleft}
    - The security of the RSA relies on the difficulty of finding d given (e, n). The problem of computing d from (e, n}) is computationally equivalent to the problem of factoring n: if one can factorise n, then he can find p and q, and hence calculate d. \\
    - p and q should differ in length by only a few digits, and both should be on the order of 100-200 digits or even larger. n with 150 digits could be factored in about 1 year. Factoring n with 200 digits could take about 1000 years (assuming about 1012 operations per second).
  \end{flushleft}
  RSA Security
}

\flashcard{\normalfont
  The \blank{\textbf{hybrid cryptosystem}} was introduced because \blank{\textbf{symmetric ciphers}} have key management problems and can not provide non repudiation services without the involvement of a trusted third party.
}

\flashcard{\normalfont
  \textbf{Hybrid cryptosystems} are the combination of \blank{\textbf{public key ciphers}} and \blank{\textbf{symmetric ciphers}}.
}

\card{\normalfont
  What does a \textbf{hybrid cryptosystem} contain?
}{\normalfont
  A \textbf{hybrid cryptosystem} contains a public cipher for symmetric key establishment/transportation and/or for digital signature generation, and a symmetric cipher for bulk encryption.
}

\card{\normalfont
  \textbf{Hybrid cryptosystems} are used to speed thing up. This process is called \textbf{digital enveloping}. What does it entail?
}{\normalfont
  \begin{flushleft}
    - A symmetrical algorithm with a random session key (bulk encryption key) is used to encrypt the message; \\
    - A public-key algorithm is used to encrypt the random session (symmetric) key.
  \end{flushleft}
  Digital Enveloping
}

\card{\normalfont
  What are the two primary uses of \textbf{public key cryptography}?
}{\normalfont \footnotesize
  \begin{flushleft}
    1. Key establishment \\
    \begin{flushleft}
      - Key exchange (or key transportation): a generates asymmetric key and transports it to b using b’s public key. RSA can be used for key exchange. \\
      - Key agreement: Both a and b co-operate to generate a shared key.
DH is a key agreement algorithm. \\
    \end{flushleft}
    2. Digital signatures: often using RSA or DSS (Digital Signature Standard) \\
  \end{flushleft}
  Public key cryptography provides capabilities that can not be attained with symmetric cryptography, but it is too inefficient to be used alone for large text encryption.
}
