% ================================ Lecture 7 ===================================
\flashcard{\normalfont
  \blank{\textbf{Public Key Infrastructure (PKI)}} provides functions, technologies policies and services that enable practical deployment and wide-scale application of public-key cryptography (secure public key distribution).
}

\card{\normalfont
  What are the PKI supported \textbf{security services}?
}{\normalfont
  \begin{flushleft}
    - Certificate-based user/entity authentication. \\
    - Digital signing of electronic documents, emails, software for authentication, integrity and nonrepudiation protections. \\
    - Encryption, typically for symmetric key distributions.
  \end{flushleft}
  PKI Supported Security Services
}

\card{\normalfont
  What are some of the common PKI \textbf{applications}?
}{\normalfont
  \begin{flushleft}
    - Web browsers, servers and services, e.g. SSL (secure socket layer). \\
    - Virtual Private Networks (VPNs), e.g. IPSec. \\
    - Secure email services, e.g. S/MIME, PGP (pretty good privacy). \\
    - Secure file storage services, e.g. PGP. \\
    - Secure electronic transactions, e.g. SET. \\
    - Visa Master smartcards. \\
    - Copyright protection (DRM), etc. \\
  \end{flushleft}
  PKI Applications
}

\flashcard{\normalfont
  A \blank{\textbf{digital certificate}} is a statement which certifies that this public key belongs to a this identity, and that the owner with this identity possesses the corresponding private key.
}

\card{\normalfont
  What are the two \textbf{trust models} of PKI?
}{\normalfont
  \begin{flushleft}
    SPKI (Simple PKI): a bottom up approach \\
    - Uses Web-of-trust model. \\
    - Public keys are signed/certified by friends or friends’ friends. \\
    - You are supposed to trust some of the friends. \\
    - Used in the PGP (pretty good privacy) solution. \\
    X509 PKI: a top-down approach \\
    - Public keys are signed/certified by trusted authorities, Certification Authorities (CAs). \\
    - A CA or CA hierarchy digitally sign keys in a top-down manner.
  \end{flushleft}
  PKI Trust Models
}

\card{\normalfont
  Certification is a secure and scalable way of distributing public keys. What does a \textbf{digital certificate} do?
}{\normalfont
  \begin{flushleft}
    - Binds an entity’s public key (+ one/more attributes) to its identity (the entity = person, hardware device, software process). \\
    - Is digitally signed by the CA so you need CA’s public key to verify the certificate. \\
    - Its contents are application dependent, e.g. a certificate for secure email contains the entity’s email address; a certificate for financial purpose may contain credit card number and credit limit, etc.
  \end{flushleft}
  Digital Certificate
}

\flashcard{\normalfont
  \blank{\textbf{Certificate Revocation Lists (CRLs)}} are a  mechanism to let the world know that a certificate is no longer valid. It is a black list of revoked certificates (i.e. prematurely terminated certificates). A CRL is a data structure, digitally signed by the issuing CA, containing: the date and time of the CRL publications; the name of the issuing CA and the serial numbers of all the revoked certificates.
}

\card{\normalfont
  Why isn't using CAs straightforward?
}{\normalfont
  \begin{flushleft}
    - The issuing CA needs to keep the CRL up-to-date. \\
    - A certificate-using application should obtain the most recent CRL and ensure that the certificate serial number is not on the CRL list; in other words, a certificate is said to be valid if and only if the following verifications are positive: it has a valid CA signature; it has not expired; it is not listed in the CA’s most recent CRL. \\
    - There are some scalability issues. That’s why short expiration policies are important.
  \end{flushleft}
  CA not straightforward
}
