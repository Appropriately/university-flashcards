% =============================== Lecture 10 ===================================
\flashcard{\normalfont
  A \blank{\textbf{Virtual Private Network (VPN)}} is a security solution, making use of tunneling, encryption, authentication, and access control technologies to allow you to achieve private communication over public networks such as the Internet.
}

\flashcard{\normalfont
  \blank{\textbf{Tunneling}} includes encapsulation, transmission and decapsulation; \textbf{encapsulation} is to wrap data with a header that provides routing information allowing it to transmit across the Internet to reach its destination so as to emulate a dedicated point-to-point link.
}

\flashcard{\normalfont
  \blank{\textbf{VPN Routers/ Gateways}} are located at the corporate network perimeter; they perform tunneling, authentication, and data encryption/ decryption.
}

\card{\normalfont
  What are the two types of \textbf{VPN Router/ Gateways}?
}{\normalfont
  \begin{flushleft}
    - \textbf{Standalone VPNs} incorporate purpose-built devices. \\
    - \textbf{Integrated} implementations add VPN functionality to existing devices such as routers and firewalls. \\
    --- \textit{Router based VPNs} add encryption support to existing routers and can keep the upgrade costs of VPN low. \\
    --- \textit{Firewall based VPNs} are workable solutions for small networks with low traffic volume.
  \end{flushleft}
  Types of VPN Routers/ Gateways
}

\flashcard{\normalfont
  A \blank{\textbf{VPN Client}} is software used for remote VPN access. It creates a secure path from the remote client computer to a VPN gateway. It can be loaded onto an individual computer requesting remote access \textbf{or} a router that establishes a peer-to-peer (router-to-router) VPN connection.
}

\card{\normalfont
  During tunnel setup, the devices on each side of the tunnel agree on the details of \textbf{authentication} and  \textbf{encryption}. What do these do?
}{\normalfont
  \begin{flushleft}
    - \textbf{Authentication} is for identifying VPN users and devices and for ensuring the authenticity of data; \\
    - \textbf{Encryption} is for protecting the confidentiality of data while in transit across the Internet.
  \end{flushleft}
  VPN Authentication and Encryption
}

\card{\normalfont
  What are some of the \textbf{security risks} on the \textbf{internet}?
}{\normalfont
  \begin{flushleft}
    - Loss of privacy (packet sniffing): a perpetrator may observe confidential data as it traverses the Internet. (Confidentiality - Encryption) \\
    - Loss of data integrity: data may be modified maliciously or accidentally. (Integrity - HMAC) \\
    - Identity spoofing: impersonation. (Entity Authentication - Keyed hash token) \\
    - Denial of Service: attacks to cause computer systems to crash.
  \end{flushleft}
  Security Risks on the Internet
}

\card{\normalfont
  What are the multiple ways of implementing a VPN?
}{\normalfont
  \begin{flushleft}
    - PPTP (point-to-point tunneling protocol) is based upon PPP (point-to-point protocol). \\
    - L2TP (Layer 2 tunneling protocol) is an extension or enhanced version of PPTP; often used together with IPSec. \\
    - IPSec \\
    Although IPSec has become the facto standard for LAN-WAN-LAN VPNs, PPTP and L2TP are heavily used for single client LAN connections. Therefore, many VPN products support IPSec, PPTP and L2TP.
  \end{flushleft}
  Ways to Implement a VPN
}

\card{\normalfont
  What are the properties of \textbf{IPSec}?
}{\normalfont \small
  \begin{flushleft}
    It operates at the IP (network) layer. \\
    It provides security protection \\
    - For the transport layer, including TCP and UDP, traffic; \\
    - For all other traffic carries in the data files of the IP packet, e.g. ICMP messages; \\
    - Also for IP packets (IPv4 and IPv6) when using tunnel mode. \\
    The protection is transparent, i.e. there is no need to modify applications or transport-layer protocols to work with IPSec, and can be applied to all the application level programs.
  \end{flushleft}
  IPSec Properties
}

\flashcard{\normalfont
  \blank{\textbf{Security Association (SA)}} refers to a set of attributes negotiated between two end-points for the protection of IP traffic for it:   Authentication mechanism (Default HMAC), Encryption algorithm and Algorithm mode (default DES-CBC), A shared session key, Initialisation Vector (IV), etc. \\
  It is \textit{unidirectional}, so for two-way secure exchange two \blank{SAs} are needed. \\
  Is uniquely identified by:\\
  - A random 32-bit value SPI (secure parameter index);\\
  - Destination (tunnel ending point) IP address; \\
  - An identifier of the security protocol (AH or ESP).
}

\card{\normalfont
  What are the two types of \textbf{key management}?
}{\normalfont \footnotesize
  \begin{flushleft}
    1. Manual management: \\
    - Manually configure keying material and SA data for each system; \\
    - Practical in small, static environments; does not scale well. \\
    2. Automated key management \\
    a. ISAKMP (Internet SA key Management Protocol): Defines procedures and packet formats to establish, negotiate, modify and delete SA. \\
    b. IKE (Internet Key Exchange): Provides facilities to negotiate and derive keying material for establishing a session key
    - DH-DSA: using Diffie-Hellman (DH) key agreement for deriving key material between peers on a public network, and DSA to sign the DH exchanges to counter the man-in-the-middle attack.
    - Public key cryptography: using recipient’s public key for secure session key transportation.
  \end{flushleft}
  Key Management Types
}

\card{\normalfont
  What are the three different types of authentication methods that the ISAKMP/IKE supports?
}{\normalfont
  \begin{flushleft}
    1. Symmetric key cryptography: The same key is pre-installed on each host. The peers authenticate each other by computing and sending a keyed hash of data that includes the pre-shared keys.\\
    2. Public key cryptography\\
    3. Digital Signature: Each device signs some data contributed by the other entity; This method is similar to scheme two, except that it does provide non-repudiation; Both RSA and DSS are supported.
  \end{flushleft}
  ISAKMP/IKE Authentication Methods
}

\card{\normalfont
  What are the properties of the ISAKMP/IKE public key cryptography authentication method?
}{\normalfont \footnotesize
  \begin{flushleft}
    - Each party generates a pseudo-random number (nonce) and encrypts it and its ID using the other party’s public key;\\
    - The ability to decrypt the data with the local private key authenticates the parties to each other. \\
    - The method requires the ability to generate random numbers, and perform public-key encryption/decryption; \\
    - It does not provide non-repudiation (as in scheme one). \\
    - Currently, only RSA algorithm is supported.
  \end{flushleft}
  SAKMP/IKE Public Key Cryptography Authentication Method
}

\flashcard{\normalfont
  Once SA(s) is negotiated and session key established, packets are forwarded using \blank{\textbf{traffic protocols}}, \blank{\textbf{AH}} and/or \blank{\textbf{ESP}}.
}

\flashcard{\normalfont
  IPSec (AH and ESP) may be employed in one of the two ways: \blank{\textbf{transport}} and \blank{\textbf{tunnel}} modes (or a combination of them).
}

\card{\normalfont
  Multiple SAs may be combines into an SA bundle. An SA can only implement either AH or ESP. In what cases would one want to combine more SAs into a bundle?
}{\normalfont
  \begin{flushleft}
    - To have both services; and/or \\
    - Different flows in one communication path requires different services.
  \end{flushleft}
  Reasons for Combining Multiple SAs into a Bundle
}

\card{\normalfont
  In what ways can SAs be combined into a bundle?
}{\normalfont
  \begin{flushleft}
    1. Transport Adjacency: \\
    - Apply ESP in transport mode without authentication; \\
    - Apply AH in transport mode. \\
    2. Iterated Tunneling (multiple nested tunnels): \\
    - Use multiple IPSec services through IP tunneling; multiple SAs in one bundle may terminate at different or same endpoint.
  \end{flushleft}
  Ways of Combining SAs into a Bundle
}

\card{\normalfont
  Each of the IPSec traffic protocols defines a new set of headers to be added to IP datagrams. Which are those?
}{\normalfont \footnotesize
  \begin{flushleft}
    - Authentication Header (AH) provides: Data origin authentication, Data integrity, Anti-replay, Does not provide confidentiality protection. \\
    - Encapsulating security payload (ESP) provides: Confidentiality (encryption) protection, Partial traffic flow confidentiality (Optional services: Data origin authentication, Data integrity, Anti-replay). \\
    - Uses keyed-hash function, HMAC, for data integrity and authentication protection (no non-repudiation protection): \\
    - Uses bulk encryption algorithms, 3-key triple DES, AES, IDEA, CAST, Blowfish and RC5, for confidentiality protection.
  \end{flushleft}
  IPSec Traffic Protocols
}

\card{\normalfont
  What is the format of the \textbf{Authentication Header (AH)} of \textbf{IPSec}?
}{\normalfont \small
  \begin{flushleft}
    - NextHeader: specifies the type of header immediately following the Authentication Header. \\
    - PayloadLength: the length of AH in 4-byte unit, minus ‘2’. \\
    - Reserved: not used for now, (set to 0). \\
    - SPI (Security Parameter Index): identifies an SA. \\
    - SequenceNumber: contains a monotonically increasing counter to protect against replay. \\
    - AuthenticationData: contains the message authentication code (MAC) for this packet (typically 96 bits long).
  \end{flushleft}
  IPSec's AH Format
}

\card{\normalfont
  What is the MAC computation?
}{\normalfont \footnotesize
  \begin{flushleft}
    The default MAC algorithm is HMAC built on keyed one-way hash function (MD5 or SHA-256 - which is detailed in the SA). It is truncated to the first 96 bits. It is stored in the AH AuthenticationData field. \\
    The following rules are applied to IP Headers (transport mode) and New IP Headers (tunnel mode) when computing the MAC:\\
    - Mutable IP header fields are zeroed prior to MAC calculation. All other (immutable) fields are included. \\
    - The AH AuthenticationData field is zeroed. All other AH header fields are included. \\
    - The entire upper-level protocol data are included.
  \end{flushleft}
  MAC Computation
}

\card{\normalfont
  What are the \textbf{Integrity and Authentication Services}?
}{\normalfont \footnotesize
  \begin{flushleft}
    Outbound packet processing (by sender): \\
    - SA lookup \\
    - Sequence number generation - must not cycle for anti-replay \\
    - MAC calculation \\
    Inbound packet processing (by receiver): \\
    - Re-assembly (if IP packet has been fragmented) \\
    - SA lookup \\
    - Sequence number verification \\
    - MAC verification: Computes MAC and verifies that it is the same as the MAC included in AuthenticationData field.
  \end{flushleft}
  Integrity and Authentication Services
}

\flashcard{\normalfont
  \blank{\textbf{Replay}} retransmits a packet to the intended destination.
}

\card{\normalfont
  What does \textbf{anti-replay} do?
}{\normalfont \footnotesize
  \begin{flushleft}
    - The SequenceNumber field of an AH is used to thwart replay attacks. \\
    - For a new SA, sequence number is initialised as 1 for the 1st packet, and increased by 1 for each outgoing packet (up to 232 - 1). If this limit is reached, then a new SA with a new key should be negotiated. \\
    - IP service is connectionless and unreliable, but IPSec requires the receiver to implement a (default) window of size W = 64 to track the out-of-order packets received, and to ensure that old or duplicated/replayed packets are discarded.
  \end{flushleft}
  Anti-replay
}

\card{\normalfont
  How does \textbf{anti-replay} work?
}{\normalfont \small
  \begin{flushleft}
    - A window size, W, specifies number of out-of-order packets that are tracked. The right edge of the window shows the highest sequence number, N, of the packet received so far. \\
    - For packet with sequence numbers in the range from N - W + 1 to N: if MAC is correct, then mark it; otherwise, drop. \\
    - If a received packet is to the right of the window and is correctly authenticated, mark the packet and advance the window. If a received packet is to the left window, drop the packet.
  \end{flushleft}
  Anti-replay and how it works
}

\card{\normalfont
  What is the format of the \textbf{Encapsulating Security Payload (ESP)} of \textbf{IPSec}?
}{\normalfont \footnotesize
  \begin{flushleft}
    - PayloadData: is a transport level segment, e.g. TCP segment, (transport mode), or IP packet (tunnel mode) that is protected by encryption. \\
    - Padding: to expand the plaintext (consisted of PayloadData, Padding, PadLth, NextHdr) to the required length and to provide partial traffic flow confidentiality. \\
    - PadLth: indicates the number of pad bytes immediately preceding it. \\
    - NextHdr: identifies the type of data contained in the PayloadData field by identifying the first header in that payload. \\
    - AuthenticationData: contains MAC computed over the ESP packet minus it.
    Other headers are the same as in AH.
  \end{flushleft}
  IPSec's Encapsulating Security Payload (ESP) Format
}

\card{\normalfont
  How does IPSec's ESP outbound packet processing look like?
}{\normalfont
  \begin{flushleft}
    - SA lookup \\
    - Packet encryption \\
    --- Encapsulate relevant data into the ESP payload field. \\
    --- Add any necessary padding. \\
    --- Encrypts the result (PayloadData, Padding, PadLength, and NextHeader) using the key, encryption algorithm indicated by the SA. \\
    - Sequence number generation. \\
    - MAC calculation (if authentication is selected by the SA).
  \end{flushleft}
  IPSec's ESP Outbound Packet Processing
}

\card{\normalfont
  How does IPSec's ESP inbound packet processing look like?
}{\normalfont
  \begin{flushleft}
    - Re-assembly (if IP packet has been fragmented by the routers en route) \\
    - SA lookup \\
    - Sequence number verification \\
    - MAC verification \\
    - Packet decryption: \\
    --- Decrypt the relevant data. \\
    --- Process any padding as specified in the encryption algorithm specification. \\
    --- Reconstruct the original IP datagram.
  \end{flushleft}
  IPSec'sESP Inbound Packet Processing
}
