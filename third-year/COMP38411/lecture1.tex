% ================================ Lecture 1 ===================================

\card{\normalfont
  What does \textbf{network security} consist of?
}{\normalfont
 \begin{flushleft}
   - Security problems and countermeasures in the transmission of information. \\
   - Security problems and countermeasures in networked computer systems.
 \end{flushleft}
 Network Security
}

\card{\normalfont
  What are some examples of \textbf{security threats}?
}{\normalfont
  \begin{flushleft}
    - Disclosure \\
    - Deception \\
    - Disruption \\
    - Attacks via Malware \\
    - Hacking as a Service \\
  \end{flushleft}
  Security Threats
}

\flashcard{\normalfont
  \blank{\textbf{Disclosure}} is the release of message contents to any person or process not possessing the appropriate cryptographic key (e.g. \blank{snooping}, \blank{sniffing}).
}

\card{\normalfont
  What is meant by \textbf{non repudiation}?
}{\normalfont
  \textbf{Non repudiation} means that the signer can not falsely deny that they have generated the signature.
}

\flashcard{\normalfont
  \blank{\textbf{Deception}} examples include:
  \begin{flushleft}
    - Interception \\
    - Modification \\
    - Spoofing \\
    - Repudiation of origin \\
    - Denial of receipt \\
  \end{flushleft}
  Security Threats
}

\flashcard{\normalfont
  \blank{\textbf{Disruption}} examples include:
  \begin{flushleft}
    - Modification \\
    - Delay \\
    - Denial of Service (DoS) \\
  \end{flushleft}
  Security Threats
}

\card{\normalfont
  What are some examples of \textbf{malware types} from Attacks Via Malware?
}{\normalfont
  \begin{flushleft}
    - Worms
    - Viruses
    - Trojan
  \end{flushleft}
  Malware Types
}

\card{\normalfont
  What are some \textbf{security problems and challenges}?
}{\normalfont \small
  \begin{flushleft}
    - Naive users: lack of security awareness \\
    - Inadequate management procedures \\
        - Insecure system set-up and configuration \\
        - Lack of proper policy making, implementation and enforcement procedures \\
    - Global networks without national boundaries \\
    - Heterogeneous devices, e.g. laptops, iPhones and PDAs, with universal connections \\
    - Wireless and open channels \\
    - Anonymous nature of many internet based services \\
  \end{flushleft}
  Security Problems and Challenges
}

\flashcard{\normalfont
  Securing information through CIA: \\
  \begin{flushleft}
    - \blank{\textbf{Confidentiality}}: keeping data and resources hidden \\
    - \blank{\textbf{Integrity}}: data \blank{integrity} (making sure data is authentic) and origin \blank{integrity} (authentication) \\
    - \blank{\textbf{Availability}}: ensuring data/service is available to authorised users.
  \end{flushleft}
  CIA
}

\card{\normalfont
  What is the \textbf{life cycle} of \textbf{achieving security}?
}{\normalfont \small
  \begin{flushleft}
    - Threats analysis and identification: decide \textbf{what to protect} \\
    - Policy specification (defining security goals): define \textbf{what is, and is not, allowed} \\
    - Design and implementation (enforcing policies to achieve security goal): decide \textbf{how to protect} in order to \textbf{satisfy the specification} with technical and procedural measures \\
    - Operation and maintenance (security assurance): assess \textbf{how well} the implementation has \textbf{achieved} its security \textbf{goal}
  \end{flushleft}
  Life Cycle of Achieving Security
}

\card{\normalfont
  What does the \textbf{threat analysis} do?
}{\normalfont
  \begin{flushleft}
    - Identify assets, threats and vulnerabilities \\
    - Assess the levels of risks on the assets based upon: \\
    --- Values of assets \\
    --- Threats to assets and their importance (Vulnerabilities and likelihood of exploitation) \\
    --- Not all threats are worth defeating (cost vs benefit) \\
    - This may be carried out by using an Attack Tree \\
    - Cost-benefit analysis: is it cheaper to prevent (using security mechanisms), recover (e.g. using restoration from backup), or just ignore?
  \end{flushleft}
  Threat Analysis
}

\flashcard{\normalfont
  An \blank{\textbf{Attack Tree (Threat Tree)}} is a conceptual diagram showing how an asset, or target, might be attacked.
}

\card{\normalfont
  What does an \textbf{Attack Tree} consists of?
}{\normalfont
  \begin{flushleft}
    - root node: the attack goal \\
    - children nodes: conditions which must be satisfied to make the direct parent node true \\
    - leaf nodes
  \end{flushleft}
  Attack Tree
}

\card{\normalfont
  What are the two conditions in an Attack Tree?
}{\normalfont
  \begin{flushleft}
    - OR: represents alternative attack methods or avenues in the attack \\
    - AND: represents multiple steps in launching an attack
  \end{flushleft}
  Attack Tree Conditions
}

\flashcard{\normalfont
  Each node may be given a value to indicate, e.g: \\
  \begin{flushleft}
    - \blank{Likelihood} that an attacker will mount the attack, or \blank{probability} of succeeding the attack \\
    - \blank{Cost} in succeeding the attack, in terms of monetary cost, or time taken to accomplish the attack, etc. \\
  \end{flushleft}
  Like this you could identify and make a decision as to what, where and how to protect your asset.
}

\flashcard{\normalfont
  In order to produce an \textbf{Attack Tree}, one needs to identify an \blank{\textbf{attack goal}} and identify all the possible \blank{\textbf{attack methods}} or \blank{\textbf{avenues}} to achieve it.
}

\flashcard{\normalfont
  \blank{\textbf{Security measures}} are methods, protocols, tools or procedures used to address the risks identified (or to enforce a security policy).
}

\flashcard{\normalfont
  \blank{Prevention} is a security measure which:
  \begin{flushleft}
    - Blocks attacks by closing vulnerabilities \\
    - Reduces the level of risks by making attacks harder \\
    - Makes another target more attractive than this one \\
    - E.g. access control (firewalls), encryption, digital signatures
  \end{flushleft}
  Security Measures
}

\flashcard{\normalfont
  \blank{Detection} is a security measure which measures taken during or after the attacks, e.g. auditing and intrusion \blank{\textbf{detection}}.
}

\flashcard{\normalfont
  \blank{\textbf{Recovery}} is a security measure which stops attacks, asseses and repairs damage. It continues to function correctly even if the attacks succeed.
}

\flashcard{\normalfont
  There is a security measure consisting of \blank{\textbf{accepting}} the attack and doing \blank{\textbf{nothing}}.
}

\flashcard{\normalfont
  A \blank{\textbf{communication}} security model: emphasis is on protecting data while in transit.
}

\flashcard{\normalfont
  A \blank{\textbf{network}} security model: focus is on protecting data and services on a \blank{\textbf{network}} against external attacks or unauthorised usage. It has multi-level security measures, however, the use of mobile devices will make the boundary hard to define.
}

\flashcard{\normalfont
  An \blank{\textbf{e-commerce}} security model: the opponent is a misbehaving insider. The third party is now a trusted third party TTP, e.g an arbitrator, that offers some services. Non-repudiation services generate the evidence the arbitrator will consider when resolving a dispute.
}
