% ================================ Lecture 1 ===================================

\card{ \normalfont
  What does \textbf{network security} consist of?
}{ \normalfont
 \begin{flushleft}
   - Security problems and countermeasures in the transmission of information. \\
   - Security problems and countermeasures in networked computer systems.
 \end{flushleft}
 Network Security
}

\card{ \normalfont
  What are some examples of \textbf{security threats}?
}{ \normalfont
  \begin{flushleft}
    - Disclosure \\
    - Deception \\
    - Disruption \\
    - Attacks via Malware \\
    - Hacking as a Service \\
  \end{flushleft}
  Security Threats
}

\flashcard{ \normalfont
  \blank{\textbf{Disclosure}} is the release of message contents to any person or process not possessing the appropriate cryptographic key (e.g. \blank{snooping}, \blank{sniffing}).
}

\flashcard{ \normalfont
  \blank{\textbf{Deception}} examples include:
  \begin{flushleft}
    - Interception \\
    - Modification \\
    - Spoofing \\
    - Repudiation of origin \\
    - Denial of receipt \\
  \end{flushleft}
  Security Threats
}

\flashcard{ \normalfont
  \blank{\textbf{Disruption}} examples include:
  \begin{flushleft}
    - Modification \\
    - Delay \\
    - Denial of Service (DoS) \\
  \end{flushleft}
  Security Threats
}

\card{ \normalfont
  What are some examples of \textbf{malware types} from Attacks Via Malware?
}{ \normalfont
  \begin{flushleft}
    - Worms
    - Viruses
    - Trojan
  \end{flushleft}
  Malware Types
}

\card{ \normalfont
  What are some \textbf{security problems and challenges}?
}{\normalfont
  \begin{flushleft}
    - Naive users: lack of security awareness \\
    - Inadequate management procedures \\
        - Insecure system set-up and configuration \\
        - Lack of proper policy making, implementation and enforcement procedures \\
    - Global networks without national boundaries \\
    - Heterogeneous devices, e.g. laptops, iPhones and PDAs, with universal connections \\
    - Wireless and open channels \\
    - Anonymous nature of many internet based services \\
  \end{flushleft}
  Security Problems and Challenges
}

\flashcard{ \normalfont
  Securing information through CIA: \\
  \begin{flushleft}
    - \blank{\textbf{Confidentiality}}: keeping data and resources hidden \\
    - \blank{\textbf{Integrity}}: data \blank{integrity} (making sure data is authentic) and origin \blank{integrity} (authentication) \\
    - \blank{\textbf{Availability}}: ensuring data/service is available to authorised users.
  \end{flushleft}
  CIA
}

\card{ \normalfont
  What is the \textbf{life cycle} of \textbf{achieving security}?
}{ \normalfont
  \begin{flushleft}
    - Threats analysis and identification: decide \textbf{what to protect} \\
    - Policy specification (defining security goals): define \textbf{what is, and is not, allowed} \\
    - Design and implementation (enforcing policies to achieve security goal): decide \textbf{how to protect} in order to \textbf{satisfy the specification} with technical and procedural measures \\
    - Operation and maintenance (security assurance): assess \textbf{how well} the implementation has \textbf{achieved} its security \textbf{goal}
  \end{flushleft}
  Life Cycle of Achieving Security
}

\flashcard{ \normalfont
  An \blank{\textbf{Attack Tree (Threat Tree)}} is a conceptual diagram showing how an asset, or target, might be attacked.
}

\card{ \normalfont
  What does an \textbf{Attack Tree} consists of?
}{ \normalfont
  \begin{flushleft}
    - root node: the attack goal \\
    - children nodes: conditions which must be satisfied to make the direct parent node true \\
    - leaf nodes
  \end{flushleft}
  Attack Tree
}

\card{ \normalfont
  What are the two conditions in an Attack Tree?
}{\normalfont
  \begin{flushleft}
    - OR: represents alternative attack methods or avenues in the attack \\
    - AND: represents multiple steps in launching an attack
  \end{flushleft}
  Attack Tree Conditions
}

\flashcard{ \normalfont
  Each node may be given a value to indicate, e.g: \\
  \begin{flushleft}
    - \blank{Likelihood} that an attacker will mount the attack, or \blank{probability} of succeeding the attack \\
    - \blank{Cost} in succeeding the attack, in terms of monetary cost, or time taken to accomplish the attack, etc. \\
  \end{flushleft}
  Like this you could identify and make a decision as what, where and how to protect your asset.
}

\flashcard{ \normalfont
  \blank{\textbf{Security measures}} are a method, protocol, tool or procedure used to address the risks identified (or to enforce a security policy).
}

\flashcard{ \normalfont
  \blank{Prevention} is a security measure which:
  \begin{flushleft}
    - Block attacks by closing vulnerabilities \\
    - Reduce the level of risks by making attacks harder \\
    - Make another target more attractive than this one \\
    - E.g. access control (firewalls), encryption, digital signatures
  \end{flushleft}
  Security Measures
}

\flashcard{ \normalfont
  \blank{Detection} is a security measure which measures taken during or after the attacks, e.g. auditing and intrusion \blank{\textbf{detection}}.
}

\flashcard{\normalfont
  \blank{\textbf{Recovery}} is a security measure which stops attacks, asses and repairs damage. It continues t function correctly even if the attacks succeed.
}

\flashcard{\normalfont
  A security measure consists of \blank{\textbf{accepting}} the attack and doing \blank{\textbf{nothing}}.
}

\flashcard{ \normalfont
  A \blank{\textbf{communication}} security model: emphasis is on protecting data while in transit.
}

\flashcard{\normalfont
  A \blank{\textbf{network}} security model: focus is on protecting data and services on a \blank{\textbf{network}} against external attacks or unauthorised usage. It has multi-level security measures, however, the use of mobile devices will make the boundary hard to define.
}

\flashcard{\normalfont
  An \blank{\textbf{e-commerce}} security model: the opponent is a misbehaving insider. The third party is now a trusted third party TTP, e.g an arbitrator, that offers some services. Non-repudiation services generate the evidence the arbitrator will consider when resolving a dispute.
}


% ================================ Lecture 2 ===================================
\flashcard{ \normalfont
  \blank{\textbf{Cryptography}} is \textit{the art of keeping messages secure}. (Schneier)
}

\card{ \normalfont
  What are some of the uses of \textbf{cryptography}?
}{ \normalfont
  \begin{flushleft}
    - Confidentiality (secrecy, privacy) of data in transmission and in storage \\
    - Integrity of data in transmission and in storage \\
    - Origin authentication of data \\
    - Authentication of identity \\
    - Credentialing systems (a proof of qualifications or competence of a person) \\
    - Digital signatures \\
    - Electronic money \\
    - Threshold cryptosystems \\
    - Secure multi-party communications \\
    - Digital right management \\
    - Electronic voting, etc.\\
  \end{flushleft}
  Cryptography Uses
}

\flashcard{ \normalfont
  \blank{\textbf{Encryption}} can be used to achieve secure communications over an insecure channel.
}

\flashcard{ \normalfont
  \blank{\textbf{Symmetric}} key based encryption uses conventional ciphers. The same key is used for encryption and decryption (e.g. historical ciphers, modern ciphers).
}

\flashcard{\normalfont
  \blank{\textbf{Asymmetric}} key based encryption uses public key ciphers. Different keys are used for encryption and decryption (e.g. modern ciphers).
}

\flashcard{\normalfont
  \blank{\textbf{Plaintext}} or \blank{\textbf{cleartext}} is a message in its original form.
}

\flashcard{\normalfont
  \blank{\textbf{Ciphertext}} is a message in an encrypted form.
}

\flashcard{\normalfont
  \blank{\textbf{Encryption}} is coding a message to hide its meaning, while \blank{\textbf{decryption}} is converting an \blank{\textbf{encrypted}} message back to its original form.
}

\flashcard{\normalfont
  \blank{\textbf{Cipher}} or \blank{\textbf{Cryptosystem}} is the system that performs encryption and decryption.
}

\flashcard{\normalfont
  An \blank{\textbf{unconditional security}} is the system that cannot be defeated, no matter how much power is available by the adversary.
}

\flashcard{\normalfont
  \blank{\textbf{Computational security}} is the perceived level of computation required to defeat the system using th best known attack exceeds, by a comfortable margin, the computational resources of the hypothesized adversary (e.g. given limited computing resources, it takes very long time to break a cipher).
}

\flashcard{\normalfont
  A \blank{\textbf{provable}} security means that the difficulty of defeating the system can be shown to be essentially as difficult as solving a well-known and supposedly difficult problem (e.g. integer factorisation).
}

\flashcard{\normalfont
  \blank{\textbf{Ad hoc}} security are claims of security which generally remain questionable, where unforseen attacks remain a threat.
}

\card{\normalfont
  What are the two classical ciphers?
}{\normalfont
  \begin{flushleft}
    1. Substitution ciphers: monoalphabetic and polyalphabetic. \\
    2. Transportation ciphers.
  \end{flushleft}
  Classical Ciphers
}

\card{\normalfont
  What are the encryption techniques?
}{\normalfont
  \begin{flushleft}
    - Classical (historical) algorithms are based on substitution (based on confusion, e.g. a becomes b) and permutation/transposition (based on diffusion, e.g. abcd becomes dacb) \\
    - XOR operator \\
    - Simple non-secure ciphers: Shift cipher - caesar cipher; Vigenere Cipher, etc. \\
    - Secure cipher: One-time pad
  \end{flushleft}
  Encryption Techniques
}

\flashcard{\normalfont
  \blank{\textbf{Ceaser Cipher}} or shift cipher uses simple substitution: each letter is translated to the letter a fixed number of letters after it in the alphabet. The operation could be expressed using addition modulo 26:
  \begin{flushleft}
    - The message must be a sequence of letters, each letter is identified with a number \\
    - The key k is a number in the range 1 to 25 \\
    - Encryption/decryption involves +- k to each letter (mod 26) \\
  \end{flushleft}
  Shift Cipher
}

\flashcard{\normalfont
  \blank{\textbf{Brute force attack}} or exhaustive key search tries all possible keys.
 }
 \card{\normalfont
   What are the three characteristics that make the Brute force attack practical?
 }{\normalfont
  \begin{flushleft}
    - The encryption and decryption algorithms are in public domain \\
    - There are only 25 keys to try out \\
    - The language of the plaintext is easily recognisable (e.g. compressed text not) \\
  \end{flushleft}
  Brute force attack practical reasons
}

\flashcard{\normalfont
  The \blank{\textbf{one-time pad}} is a special variant of the stream cipher and it is truly a perfect cipher: it uses a one-time random key that is as long as the plaintext with no repetitions (only used once). If used properly, it is provably unbreakable.
}

\card{\normalfont
  What are some of the \textbf{problems} of the \textbf{one-time pad}?
}{\normalfont
  \begin{flushleft}
    - No way to remember the entirety of the random key string. \\
    - It’s too expensive for most applications as it consumes as much key material as there is traffic. \\
    - Key management is hard as the need for non-repeating keys has a problem with storing and distributing them. \\
    - Absolute synchronisation between sender and receiver is really hard to achieve.\\
  \end{flushleft}
  Problems with the one-time pad
}

\card{\normalfont
  What is the main idea of a \textbf{stream cipher}?
}{\normalfont
  To replace the random key in one-time pad by a pseudo-random sequence, generated by a cryptographic pseudo-random generator that is seeded with the key. \\
  \textit{Ciphertext = plaintext XOR keystream} \\
  This is a dangerous security property and we must never reuse the same keystream to encrypt two different messages.
}

\flashcard{\normalfont
  The \blank{\textbf{transportation (permutation)}} texhnique performs a permutation on the plaintext.
}

\card{\normalfont
  What are some common types of \textbf{attacks}?
}{\normalfont
  \begin{flushleft}
    - Try to break or crack the algorithm by exploring any flaws in it e.g. frequency analysis. \\
    - Assume attacker recognises plaintext, try to decrypt ciphertext with every possible key until…, e.g. brute force attack (Also exhaustive key search attack) \\
    - Run the algorithm on massive amounts of plaintext and find the one plaintext that encrypts to the ciphertext that it’s being analysed, e.g. dictionary attack.\\
  \end{flushleft}
  Common types of attacks
}

\card{\normalfont
  Ciphertest-only attacks? (e.g. frequency attcks)
}{\normalfont
  \begin{flushleft}
    - Attacker knows ciphertexts of several messages encrypted with the same key, plaintext is recognisable. \\
    - The goal is to find the plaintext, possibly the key \\
  \end{flushleft}
  Ciphertext-only attacks
}

\card{\normalfont
  Known-plaintext attack? (e.g. dictionary attack)
}{\normalfont
  \begin{flushleft}
    - Attacker observes <plaintext, ciphertext> pairs encrypted with the same key. \\
    - The goal is to find the key. \\
  \end{flushleft}
  Known-plaintext attack
}

\card{\normalfont
  Chosen-plaintext attack?
}{\normalfont
  \begin{flushleft}
    - Attacker can choose the plaintext and look at the paired ciphertext. \\
    - The goal is to find the key. \\
  \end{flushleft}
  Chosen-plaintext attack
}

\flashcard{\normalfont
  \blank{\textbf{Cryptanalysis attacks}} often exploit the redundancy of natural language: lossless compression before encryption removes redundancy.
}


% ================================ Lecture 3 ===================================




% ================================ Lecture 4 ===================================