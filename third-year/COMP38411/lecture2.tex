% ================================ Lecture 2 ===================================
\flashcard{\normalfont
  \blank{\textbf{Cryptography}} is \textit{the art of keeping messages secure}. (Schneier)
}

\card{\normalfont
  What are some of the uses of \textbf{cryptography}?
}{\normalfont \footnotesize
  \begin{flushleft}
    - Confidentiality (secrecy, privacy) of data in transmission and in storage \\
    - Integrity of data in transmission and in storage \\
    - Origin authentication of data \\
    - Authentication of identity \\
    - Credentialing systems (a proof of qualifications or competence of a person) \\
    - Digital signatures \\
    - Electronic money \\
    - Threshold cryptosystems \\
    - Secure multi-party communications \\
    - Digital right management \\
    - Electronic voting, etc.\\
  \end{flushleft}
  Cryptography Uses
}

\flashcard{\normalfont
  \blank{\textbf{Encryption}} can be used to achieve secure communications over an insecure channel.
}

\card{\normalfont
  What are the two types of \textbf{encryption}?
}{\normalfont
  \begin{flushleft}
    - \textbf{Symmetric} key based encryption uses conventional ciphers. The same key is used for encryption and decryption (e.g. historical ciphers, modern ciphers). \\
    - \textbf{Asymmetric} key based encryption uses public key ciphers. Different keys are used for encryption and decryption (e.g. modern ciphers).
  \end{flushleft}
  Types of Encryption
}

\flashcard{\normalfont
  \blank{\textbf{Plaintext}} or \blank{\textbf{cleartext}} is a message in its original form.
}

\flashcard{\normalfont
  \blank{\textbf{Ciphertext}} is a message in an encrypted form.
}

\flashcard{\normalfont
  \blank{\textbf{Encryption}} is coding a message to hide its meaning, while \blank{\textbf{decryption}} is converting an \blank{\textbf{encrypted}} message back to its original form.
}

\flashcard{\normalfont
  \blank{\textbf{Cipher}} or \blank{\textbf{Cryptosystem}} is the system that performs encryption and decryption.
}

\flashcard{\normalfont
  \blank{\textbf{Cryptanalysis}} attempts to find plaintext or key.
}

\flashcard{\normalfont
  An \blank{\textbf{unconditional security}} is the system that cannot be defeated, no matter how much power is available by the adversary.
}

\flashcard{\normalfont
  \blank{\textbf{Computational security}} is the perceived level of computation required to defeat the system using the best known attack exceeds, by a comfortable margin, the computational resources of the hypothesized adversary (e.g. given limited computing resources, it takes very long time to break a cipher).
}

\flashcard{\normalfont
  A \blank{\textbf{provable}} security means that the difficulty of defeating the system can be shown to be essentially as difficult as solving a well-known and supposedly difficult problem (e.g. integer factorisation).
}

\flashcard{\normalfont
  \blank{\textbf{Ad hoc}} security are claims of security which generally remain questionable, where unforseen attacks remain a threat.
}

\card{\normalfont
  What are the two classical ciphers?
}{\normalfont
  \begin{flushleft}
    1. Substitution ciphers: monoalphabetic and polyalphabetic. \\
    2. Transportation ciphers.
  \end{flushleft}
  Classical Ciphers
}

\card{\normalfont
  What are the encryption techniques?
}{\normalfont
  \begin{flushleft}
    - Classical (historical) algorithms are based on substitution (based on confusion, e.g. a becomes b) and permutation/transposition (based on diffusion, e.g. abcd becomes dacb) \\
    - XOR operator \\
    - Simple non-secure ciphers: Shift cipher - caesar cipher; Vigenere Cipher, etc. \\
    - Secure cipher: One-time pad
  \end{flushleft}
  Encryption Techniques
}

\flashcard{\normalfont \small
  \blank{\textbf{Ceaser Cipher}} or shift cipher uses simple substitution: each letter is translated to the letter a fixed number of letters after it in the alphabet. The operation could be expressed using addition modulo 26:
  \begin{flushleft}
    - The message must be a sequence of letters, each letter is identified with a number \\
    - The key k is a number in the range 1 to 25 \\
    - Encryption/decryption involves +- k to each letter (mod 26) \\
  \end{flushleft}
  Shift Cipher
}

\flashcard{\normalfont
  \blank{\textbf{Brute force attack}} or exhaustive key search tries all possible keys.
 }

 \card{\normalfont
   What are the three characteristics that make the \textbf{Brute Force Attack} practical?
 }{\normalfont
  \begin{flushleft}
    - The encryption and decryption algorithms are in public domain \\
    - There are only 25 keys to try out \\
    - The language of the plaintext is easily recognisable (e.g. compressed text not) \\
  \end{flushleft}
  Brute force attack practical reasons
}

\flashcard{\normalfont
  The \blank{\textbf{one-time pad}} is a special variant of the stream cipher and it is truly a perfect cipher: it uses a one-time random key that is as long as the plaintext with no repetitions (only used once). If used properly, it is provably unbreakable.
}

\card{\normalfont
  What are some of the \textbf{problems} with the \textbf{one-time pad}?
}{\normalfont
  \begin{flushleft}
    - No way to remember the entirety of the random key string. \\
    - It’s too expensive for most applications as it consumes as much key material as there is traffic. \\
    - Key management is hard as the need for non-repeating keys has a problem with storing and distributing them. \\
    - Absolute synchronisation between sender and receiver is really hard to achieve.\\
  \end{flushleft}
  Problems with the one-time pad
}

\card{\normalfont
  What is the main idea of a \textbf{stream cipher}?
}{\normalfont
  \begin{flushleft}
    To replace the random key in one-time pad by a pseudo-random sequence, generated by a cryptographic pseudo-random generator that is seeded with the key. \\
    \textit{Ciphertext = plaintext XOR keystream} \\
    This is a dangerous security property and we must never reuse the same keystream to encrypt two different messages.
  \end{flushleft}
  Stream Cipher
}

\flashcard{\normalfont
  The structure of \blank{\textbf{stream ciphers}} is to generate a keystream from a short key that initialises the generator.
}

\flashcard{\normalfont
  The \blank{\textbf{transportation (permutation)}} technique performs a permutation on the plaintext.
}

\card{\normalfont
  What are some common types of \textbf{attacks}?
}{\normalfont
  \begin{flushleft}
    - Try to break or crack the algorithm by exploring any flaws in it e.g. frequency analysis. \\
    - Assume attacker recognises plaintext, try to decrypt ciphertext with every possible key until…, e.g. brute force attack (Also exhaustive key search attack) \\
    - Run the algorithm on massive amounts of plaintext and find the one plaintext that encrypts to the ciphertext that it’s being analysed, e.g. dictionary attack.\\
  \end{flushleft}
  Common types of attacks
}

\card{\normalfont
  Ciphertext-only attacks? (e.g. frequency attcks)
}{\normalfont
  \begin{flushleft}
    - Attacker knows ciphertexts of several messages encrypted with the same key, plaintext is recognisable. \\
    - The goal is to find the plaintext, possibly the key \\
  \end{flushleft}
  Ciphertext-only attacks
}

\card{\normalfont
  Known-plaintext attack? (e.g. dictionary attack)
}{\normalfont
  \begin{flushleft}
    - Attacker observes <plaintext, ciphertext> pairs encrypted with the same key. \\
    - The goal is to find the key. \\
  \end{flushleft}
  Known-plaintext attack
}

\card{\normalfont
  Chosen-plaintext attack?
}{\normalfont
  \begin{flushleft}
    - Attacker can choose the plaintext and look at the paired ciphertext. \\
    - The goal is to find the key. \\
  \end{flushleft}
  Chosen-plaintext attack
}

\flashcard{\normalfont
  \blank{\textbf{Cryptanalysis attacks}} often exploit the redundancy of natural language: lossless compression before encryption removes redundancy.
}
