\flashcard{
  Median smoothing is used to \blank{suppress noise in an image}.
}

\card{What is convolution of an image and a template? What is it used for?}{
  Convolution is the process of adding each element of the image to its local neighbors, weighted by the kernel. \\ 
  It measures the similarity between image patches and the template.
}

\flashcard{
  Stretching the grey scale in a black and white image will \blank{increase the amount of contrast in the image}.
}

\flashcard{
  In order to scan-convert a triangle, we must always \blank{know the coordinates of the triangle's vertices}.
}

\card{What problem does the \textbf{Z-buffer} solve?}{
  The hidden-surface problem.
}

\flashcard{
  Homogeneous coordinates are used in Computer Graphics to provide a \blank{consistent representation for different types} of transformations.
}

% =============================== Sample Exam =============================== %

% 1)
\card{Explain the distinction between the activities of “modelling” and “rendering”, in Computer Graphics.}{
  Modelling is about constructing something to represent an idea; in CG the model is usually geometrical; rendering is the process of creating a visual representation of the model.
}

% 2)
\card{What is a limitation of using a $3\times3$ matrix to represent 3D 
transformations?}{
  A $3\times3$ matrix cannot represent a 3D translation transformation.
}

% 3)
\flashcard{
  In 3D computer graphics, the function of a camera transforms coordinates to simulate the operation of a real camera which creates 2D images of a 3D world.
}

% 4)

% 5)
\card{In viewing, what is meant by the near and far clip planes?}{
  The near and far clip planes set the forward and rear boundaries of the view volume; geometry is clipped at these boundaries.
}

% 6)
\card{Explain how you would apply the local illumination model to a triangle mesh, such that it is smoothly shaded, and correctly takes into account specular reflection.}{
  For each triangle T in mesh: Compute the average normal vectors at each vertex V of T by averaging surface normal of the triangles which share V; then process each scanline for the triangle; for the part of scanline S which covers T, compute an averaged normal for pixels at the start (P) and end (Q) of the scanline segment by averaging from the respective vertices; then step along the scanline from P to Q, and apply the full local illumination model for each pixel, smoothly interpolating the normal for each pixel between P and Q.
}

% 7)
\card{What is a normalized vector? Give an example of its use in computer graphics}{
  A vector with length one; common use: surface normal, to give the direction a triangle is facing, needed during rendering.
}

% 8)
\card{What is the purpose of the OpenGL "matrix stack"? Illustrate your answer with a practical example of its use.}{
  The matrix stack is for saving the current transformation T; the user can modify T, and then reset T to the saved value. Example of uses: drawing an overlay on the screen for a framerate counter. We stack the current view, replace it with the overlay view, then unstack and restore the usual view.
}

% 9)


% 10)
\card{Why do textures usually need to be filtered during rendering? Describe the bilinear interpolation filter.}{
  Usually texel resolution does not match pixel resolution, therefore we need to apply a filter to resolve the mis-match; the bilinear interpolation filter is used when the texel resoluion is less than the pixel resolution; each pixel receives a colour averaged from adjacent texels.
}

% 11)
\flashcard{
  The histogram of an image represents the \blank{frequency of occurrence of each grey or colour value}.
}

% 12)
\card{An image might be non-uniformly illuminated. What is the consequence of this for thresholding? How could this be corrected?}{
  Consequence is the threshold is correct in some regions, too high or too low in others. Correction is by adjusting the threshold throughout the image.
}

% 13)
\card{Explain the process of convolution as it is used in image processing. Give a simple interpretation of the meaning of the results.}{
  Convolution requires a template. This is placed at all locations on the image, overlapping image/template values are multiplied and added to give a result.  Special consideration in areas where the template extends beyond the image boundary: either ignore these locations or pad the image. The result measures the resemblance of the template with that portion of the image.
}

% 14)
\card{What principles would you follow to design the kernel to be used to recognize cars from a rear view? What would your kernel look like?}{
  Always looking for invariants – properties of the object that are the same for all instances of the object. In this case it could be the shadow under the car.
}

% 15)
\card{You’ve been asked to implement software to identify flooded areas from low-level aerial images, such as the one below. How will you do this? How will you validate your results?}{
  Task could be achieved by thresholding – would expect the answer to include a discussion of how to find a threshold. Or by using texture –you’d expect waterlogged regions to look different on a small scale to dry areas. So if you can measure local properties of an image you have a way of classifying, local properties could include variance. The classification image should be cleaned up to reduce false positives/negatives – median filter. Validation is by comparing to manually labelled images.
}

% 16)
\card{Colour data is represented by using three primitives. What are they and why were they chosen?}{
  Primitives are called RGB but they’re not really red, green, blue. They are matched to the response of the human visual system to longer, medium and shorter wavelength illuminants.
}

% 17)
\card{What data would you use to identify coloured objects in a range of light intensities?}{
  Chromaticities - either HS from HSV or normalized red and normalized green.
}

