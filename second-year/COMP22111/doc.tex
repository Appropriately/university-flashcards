% Bascically just an adaption of Todd Davies flashcards for the subjects,
% with plans to basically expand upon them

\documentclass[frontgrid]{flacards}
\usepackage{color}
% For funky database symbols
\usepackage{newlfont}
\usepackage{tabularx}
\usepackage{graphicx}

\definecolor{light-gray}{gray}{0.75}
\fboxsep=20pt

\newcommand{\frontcard}[1]{\fboxsep=2pt\textcolor{light-gray}{\colorbox{light-gray}{$#1$}}}
\newcommand{\backcard}[1]{#1}
\newcommand\tab[1][0.5cm]{\hspace*{#1}}

\newcommand{\flashcard}[1]{% create new command for cards with blanks
    \card{% call the original \card command with twice the same argument (#1)
        \let\blank\frontcard% but let \blank behave like \frontcard the first time
        #1
    }{%
        \let\blank\backcard% and like \backcard the second time
        #1
    }%
}

\begin{document}

\pagesetup{2}{4} 

\flashcard{\blank{Abstraction}: hiding the complexity of the design.}

\flashcard{The \blank{datapath} realises the data operations required by the FSM.}

\flashcard{The \blank{control} issues signals to control the operation of the datapath.}

\card{
  A Reduced Instruction Set Computer (RISC) has three instruction types. What are they, and in brief what do they do? \textit{Hint: Stump is a RISC computer}
}{
  Register to Register: Instructions that perform operations on values inside of registers.\\
  Load/Store: The only operations that will be on memory.\\
  Branch: Operations that change the value of the PC depending on a certain condition.
}

\flashcard{You use bit \blank{12} to determine whether an instruction is Type 1 or 2.}

\flashcard{Shifter operations are applied to the \blank{srcA} register in \blank{type 1} operations.}

\card{
  In Verilog, there are four options for a bit to be. 0, 1, X and Z.\\In brief, what does each one mean?
}{
  1 and 0 are True and False respectively.\\A value of X is unknown logic. \\A value of Z would be high impedance.
}

\card{
  What type of circuits do you associate the use of\\
  i)  blocking statements, and\\
  ii) non-blocking statements\\
  in Verilog code?
}{
  Blocking statements use $=$, and are associated with combinatorial logic and execute one at a time.\\
  Non-blocking statements use $<=$, and are associated with sequential systems that use a clock and actions are executed concurrently. 
}

\card{
  Do you have to use the default case in Verilog? When would you use it?
}{
  Not necessarily. It is useful to cover cases not listed in the case statement, trapping any invalid states you shouldn't be in.
}

\card{
  What is the difference between a Verilog \textbf{task} and a Verilog \textbf{function}?
}{
  A task can change any number of outputs whereas a function can update only one.
}

\card{
  Where would you locate a \textbf{task} or \textbf{function} in your Verilog code?
}{
  It needs to be within the module but outside any always/initial blocks.
}

\card{
  Why does Stump's R0 exist? Give some examples illustrating its use.
}{
  Allows you to enable further operations to be implemented that are not part of the ISA.\\
  \begin{tabularx}{0.32\textwidth}{l X}
    - & MOV: ADD  R3, R2, R0\\
    - & NOP: ADD  R0, R0, R0\\
    - & CMP: SUBS R0, R3, R4\\
	\end{tabularx}
}

\card{
  What is a load/store architecture?
}{
  Data within registers will only be operated on, with results being written back to registers. LD/ST operations will be included for memory operations.
}

\card{
  How can pipelining accelerate performance of a processing system?
}{
  By introducing parallelism at a small cost. It allows more instruction throughput at a given clock rate than would traditionally be possible.
  % TO DO
}

\end{document} 
