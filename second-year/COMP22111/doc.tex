% Bascically just an adaption of Todd Davies flashcards for the subjects,
% with plans to basically expand upon them

\documentclass[frontgrid]{flacards}
\usepackage{color}
% For funky database symbols
\usepackage{newlfont}
\usepackage{tabularx}
\usepackage{graphicx}

\definecolor{light-gray}{gray}{0.75}

\newcommand{\frontcard}[1]{\textcolor{light-gray}{\colorbox{light-gray}{$#1$}}}
\newcommand{\backcard}[1]{#1} 

\newcommand{\flashcard}[1]{% create new command for cards with blanks
    \card{% call the original \card command with twice the same argument (#1)
        \let\blank\frontcard% but let \blank behave like \frontcard the first time
        #1
    }{%
        \let\blank\backcard% and like \backcard the second time
        #1
    }%
}

\begin{document}

\pagesetup{2}{4} 

\flashcard{\blank{Abstraction}: hiding the complexity of the design.}

\flashcard{The \blank{datapath} realises the data operations required by the FSM.}

\flashcard{The \blank{control} issues signals to control the operation of the datapath.}

\card{
  A Reduced Instruction Set Computer (RISC) has three instruction types. What are they, and in brief what do they do?
}{
  Register to Register: Instructions that perform operations on values inside of registers.\\
  Load/Store: The only operations that will be on memory.\\
  Branch: Operations that change the value of the PC depending on a certain condition.
}

\flashcard{Shifter operations are applied to the \blank{srcA} register in \blank{type 1} operations.}

\card{
  In Verilog, there are four options for a bit to be. 0, 1, X and Z.\\In brief, what does each one mean?
}{
  1 and 0 are True and False respectively.\\A value of X is unknown logic. \\A value of Z would be high impedance.
}

\card{
  What type of circuits do you associate the use of\\
  i)  blocking statements, and\\
  ii) non-blocking statements\\
  in Verilog code?
}{
  Blocking statements use $=$, and are associated with combinatorial logic and execute one at a time.\\
  Non-blocking statements use $<=$, and are associated with sequential systems that use a clock and actions are executed concurrently. 
}

\card{
  How can pipelining accelerate performance of a processing system?
}{
  % TO DO
}

\end{document} 
