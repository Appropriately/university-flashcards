\documentclass[frontgrid]{flacards}
\usepackage{color}
\usepackage{mathtools}
\usepackage{amsmath}
\usepackage{amssymb}

\definecolor{light-gray}{gray}{0.75}

\newcommand{\frontcard}[1]{\textcolor{light-gray}{\colorbox{light-gray}{$#1$}}}
\newcommand{\backcard}[1]{#1} 
\newcommand\tab[1][0.5cm]{\hspace*{#1}}

\newcommand{\flashcard}[1]{% create new command for cards with blanks
    \card{% call the original \card command with twice the same argument (#1)
        \let\blank\frontcard% but let \blank behave like \frontcard the first time
        #1
    }{%
        \let\blank\backcard% and like \backcard the second time
        #1
    }%
}

\begin{document}

\pagesetup{2}{4} 

\card{
  What does $O(<expr>)$ mean?
}{
  The complexity (i.e. running time/space) is bounded by the $<expr>$.
}

\card{
  What does $\Theta(<expr>)$ mean?
}{
  The complexity (i.e. space/running time) has the complexity proportional to
  $<expr>$.
}

\card{
  What does $\Omega(<expr>)$ mean?
}{
  The complexity (i.e. running time/space) is \textit{at least} by the $<expr>$.
}

\card{
  What are the best, average and worst case complexities of \textbf{Bubble Sort}?
}{
  Best: $O(n)$,\\ Average: $O(n^2)$,\\ Worst: $O(n^2)$
}

\card{
  What are the best, average and worst case complexities of \textbf{Merge Sort}?
}{
  Best: $O(nlog\log_2n)$,\\ Average: $O(nlog\log_2n)$,\\ Worst: $O(nlog\log_2n)$
}

\card{
  Give pseudo code for merging 2 sorted lists, as part of merge sort.
}{
  \begin{flushleft}
    Merge($L_1$, $L_2$) \\
    \tab if $L_1 = []$ return $L_2$ \\
    \tab if $L_2 = []$ return $L_1$ \\
    \tab $x_1 = L_1 [0]$ \\
    \tab $x_2 = L_2 [0]$ \\
    \tab $L'_1 = L_1 [1 : \vert L_1 \vert - 1]$ \\
    \tab $L'_2 = L_2 [1 : \vert L_2 \vert - 1]$ \\
    \tab if $x_1 \leq x_2$ \\
    \tab \tab return $[x_1]+$Merge($L'_1$ ,$L_2$) \\
    \tab return $[x_2]+$Merge($L_1$ ,$L'_2$) 
  \end{flushleft}
  \textit{Merge two sorted lists}
}

\card{
  Give pseudo code for MergeSort(L).
}{
  \begin{flushleft}
    MergeSort(L) \\
    \tab if $\vert L \vert \leq 1$ \\
    \tab \tab return L \\
    \tab Split L into roughly equal halves, $L_l$ and $L_r$ \\
    \tab return Merge(MergeSort($L_l$),MergeSort($L_r$))
  \end{flushleft}
  \textit{MergeSort(L)}
}

\card{
  What are the best, average and worst case complexities of \textbf{Quick Sort}?
}{
  Best: $O(nlog\log_2n)$,\\ Average: $O(nlog\log_2n)$,\\ Worst: $O(n^2)$
}

\card{
  What would the pseudo code be for Quick Sort?
}{
  \begin{flushleft}
    quicksort(L) \\
    \tab if length of $L \leq 1$ \\
    \tab\tab return L \\
    \tab remove the first element, x, from L \\
    \tab $L_\leq :=$ elements of L less than or equal to x \\
    \tab $L_> :=$ elements of L greater than x \\
    \tab $L_l :=$ quicksort($L_\leq$) \\
    \tab $L_r :=$ quicksort($L_>$) \\ 
    \tab return $L_l$ + [x] + $L_r$ \\
  \end{flushleft}
  \textit{Quick Sort}
}

\card{ 
  Say that the input represents a positive integer, $x$, what is the size of $n$?
 }{
  $\left \lfloor \log_b x \right \rfloor + 1$
  Where $b$ is the number representation, usually binary (so 2).
 }

\card{
  What does it mean by $O(1)$?
}{
  It takes a constant time, no matter the amount of data, to perform the operation.
}

\card{
  What is the minimum time for any sorting algorithm that uses only number comparisons?
}{
  $n\log_2n$
}

\card{
  What would the pseudo code be for Euclid's algorithm?
}{
  \begin{flushleft}
    // Assume $a>=b$ \\
    hcf(a,b) \\
    \tab if $b = 0$ \\
    \tab\tab return a \\
    \tab $r=amodb$ \\
    \tab return hcf(b,r)
  \end{flushleft}
  \textit{Euclid's algorithm}
}

\card{
  What would the pseudo code be for Fast Modular Exponentiation?
}{
  \begin{flushleft}
    fme(a,b,k) \\
    \tab $d=a$ \\
    \tab $e=b$ \\
    \tab $s=1$ \\
    \tab While $e>0$ \\
    \tab\tab if e is odd \\
    \tab\tab\tab $s=(s.d)mod k$ \\
    \tab\tab $d = d^2 mod k$ \\
    \tab\tab $e = \lfloor e / 2 \rfloor$ \\
    \tab return s \\
  \end{flushleft}
  \textit{Fast Modular Exponentiation}
}

\card{
  Give pseudo code for calculating the Discrete Logarithm?
}{
  \begin{flushleft}
    discreteLog(p,g,b) \\
    \tab $x := 1$ \\
    \tab While x is less than p \\
    \tab \tab $y = g^x mod p$ \\ 
    \tab \tab if $x=b$ \\ 
    \tab \tab \tab return x \\ 
    \tab \tab $x++$ \\
  \end{flushleft}
  \textit{Calculates Discrete Logarithm}
}

\card{
  Consider the equation $a^x = y mod p$. If $a$ is a primitive root modulo $p$, then for every $y (1\leq y < p)$, such an $x (1\leq x < p)$ exists. What is $x$?
}{
  X is the \textbf{discrete logarithm} of $y$ with base $a$, modulo $p$.
}

\flashcard{The \blank{discrete logarithm} is the inverse of exponentiation.}

\card{
  Why can the private key not, in practice, be recovered from the public key when $p$ is large?
}{
  To calculate a public key, $y$, a private key, $x$ is needed. The equation for modular exponentiation can be used: $y = g^x mod p$ \\
  It is considered a one-way function - easy to compute, hard to invert. For a large $p$, the only way to figure out the private key would be to use brute force, which would take a large amount of time.
}

\card{
  What is one way you can argue correctness of Euclid's algorithm?
}{
  Let $r = a mod b$. $hcf(a,b)=hcf(b,r)$ because all factors of a and b are also factors of b and r and vice versa. If they have the same factors, they have the same highest common factor.
}

\card{
  What would half the correctness proof be for Euclid's algorithm?
}{
  As $r=a mod b$, $\exists q$ such that $a=bq+r$, $\therefore r = a-bq$. \\
  Suppose x is a factor of a and b, then $\exists y and z$ such that $a=xy$, $b=xz$. \\
  Hence: $r=xy-xzq$, $r=x(y-zq)$. \\ 
  $\therefore x$ is a factor of r (and also of b and r).
}

\flashcard{
  $(a.b)modk=$ \blank{$(amodk.bmodk)modk$}
}

\card{
  Let $p$ be a prime number. What is meant by a primitive root modulo $p$?
}{
  The numbers $r_x$ between 1 and $p-1$ that, when raised by the numbers between 1 and $p-1$ compute all the numbers between 1 and $p-1$ in some order with no repetitions.
}

\card{
  What does saying that algorithm A runs in time g mean?
}{
  Given an input of size n, the number of operations executed by A is bounded above by g(n).
}
\end{document} 
