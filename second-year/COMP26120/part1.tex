\documentclass[frontgrid]{flacards}
\usepackage{color}
\usepackage{mathtools}
\usepackage{amsmath}
\usepackage{amssymb}

\definecolor{light-gray}{gray}{0.75}

\newcommand{\frontcard}[1]{\textcolor{light-gray}{\colorbox{light-gray}{$#1$}}}
\newcommand{\backcard}[1]{#1} 

\newcommand{\flashcard}[1]{% create new command for cards with blanks
    \card{% call the original \card command with twice the same argument (#1)
        \let\blank\frontcard% but let \blank behave like \frontcard the first time
        #1
    }{%
        \let\blank\backcard% and like \backcard the second time
        #1
    }%
}

\begin{document}

\pagesetup{2}{4} 

\card{
  What does $O(<expr>)$ mean?
}{
  The complexity (i.e. running time/space) is bounded by the $<expr>$.
}

\card{
  What does $\Theta(<expr>)$ mean?
}{
  The complexity (i.e. space/running time) has the complexity proportional to
  $<expr>$.
}

\card{
  What does $\Omega(<expr>)$ mean?
}{
  The complexity (i.e. running time/space) is \textit{at least} by the $<expr>$.
}

\card{ 
  Say that the input represents a positive integer, $x$, what is the size of $n$?
 }{
  $\left \lfloor \log_b x \right \rfloor + 1$
  Where $b$ is the number representation, usually binary (so 2).
 }

\card{
  What does it mean by $O(1)$?
}{
  It takes a constant time, no matter the amount of data, to perform the operation.
}

\card{
  What is the minimum time for any sorting algorithm that uses only number comparisons.
}{
  $n\log_2n$
}

\card{
  How do you argue correctness of Euclid's algorithm?
}{
  Let $r = a mod b$. $hcf(a,b)=hcf(b,r)$ because all factors of a and b are also factors of b and r and vice versa. If they have the same factors, they have the same highest common factor.
}

\card{
  What would half the correctness proof be for Euclid's algorithm?
}{
  As $r=a mod b$, $\exists q$ such that $a=bq+r$, $\therefore r = a-bq$.
  Suppose x is a factor of a and b, then $\exists y and z$ such that $a=xy$, $b=xz$. Hence:
  $r=xy-xzq$, $r=x(y-zq)$. $\therefore x$ is a factor of r (and also of b and r).
}

\flashcard{
  $(a.b)modk=$ \blank{$(amodk.bmodk)modk$}
}

\card{
  Let $p$ be a prime number. What is meant by a primitive root modulo $p$?
}{
  The numbers $r_x$ between 1 and $p-1$ that, when raised by the numbers between 1 and $p-1$ compute all the numbers between 1 and $p-1$ in some order with no repetitions.
}

\card{
  What does saying that algorithm A runs in time g mean?
}{
  Given an input of size n, the number of operations executed by A is bounded above by g(n).
}
\end{document} 
