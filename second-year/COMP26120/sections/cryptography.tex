

\card{
  What would the pseudo code be for Euclid's algorithm?
}{
  \begin{flushleft}
    // Assume $a>=b$ \\
    hcf(a,b) \\
    \tab if $b = 0$ \\
    \tab\tab return a \\
    \tab $r=amodb$ \\
    \tab return hcf(b,r)
  \end{flushleft}
  \textit{Euclid's algorithm}
}

\card{
  What would the pseudo code be for Fast Modular Exponentiation?
}{
  \begin{flushleft}
    fme(a,b,k) \\
    \tab $d=a$ \\
    \tab $e=b$ \\
    \tab $s=1$ \\
    \tab While $e>0$ \\
    \tab\tab if e is odd \\
    \tab\tab\tab $s=(s.d)mod k$ \\
    \tab\tab $d = d^2 mod k$ \\
    \tab\tab $e = \lfloor e / 2 \rfloor$ \\
    \tab return s \\
  \end{flushleft}
  \textit{Fast Modular Exponentiation}
}

\card{
  What are some of the advantages of ElGamal encryption?
}{
     Sender Verification \\
     Private key remains with owner \\
     Public key is freely distributable \\
     No secret channel needed at any point \\
     No need for pre-shared keys
}

\card{
  What is the basic procedure for an encryption and decryption using publik key
  cryptography if Alice wants to send a message to Bob?
}{
  Alice generates a private random integer a and Bob generates a private
  random integer b \\
  Alice generates her public value $g^a \bmod{p}$ \\
  Bob generates his public value $g^b \bmod{p}$ \\
  Alice computes $g^{ab} = (g^a)^b \bmod{p}$ \\
  Bob computes $g^{ba} = (g^b)^a \bmod{p}$ \\
  Now they have a shared secret $k$ since $k = g^{ab} = g^{ba}$
}

\card{
  Describe public key generation in ElGamal encryption using p as the Prime
  Modulus and g as the Primitive root (as described in the COMP26120 lab)
}{
  Generate a large $p$ and a $g$ in $1 \leq g < p$ \\
  Generate a random integer $a$ in $1 \leq a \leq p - 2$ \\
  Compute $g^a \bmod{p}$. The public key is
  $$(p, g, g^a)$$
  The private key is $a$
}

\card{
  Describe the encryption procedure used in the ElGamal cryptosystem given that
  person B wants to send message M to preson A
}{
  Obtain A's public key $(p, g, g^a)$ \\
  Represent the message M as integers in the range ${0, ..., p - 1}$ \\
  Select a random integer k from $1 \leq k \leq p - 2$ \\
  Compute $\gamma = g^k \bmod{p}$ and $\delta = m \cdot (g^a)^k$ \\
  Send ciphertext $c = (\gamma, \delta)$ to A
}

\card{
  Describe the decryption process used in the ElGamal cryptosystem given that
  person A has received cyphertext $(\gamma, \delta)$ from person B, encrypted
  encrypted using the public key $(p, g, g^a)$
}{
  Use private key a to compute $(\gamma^{p-1-a}) \bmod{p}$ \\
  NOTE THAT: $(\gamma^{p-1-a}) = \gamma^{-a} = g^{-ak}$ \\
  Recover the message M by computing $(\gamma^{-a} \cdot \delta \bmod{p})$ \\
  Note that this evaluates to $(g^{-ak} \cdot g^{ak} \cdot M \bmod{p})$ or
  $1 \cdot M \bmod{p}$
}

\card{
  Consider the equation $a^x = y \bmod{p}$. If $a$ is a primitive root of modulo $p$, then for every $y (1\leq y < p)$, such an $x (1\leq x < p)$ exists. What is $x$?
}{
  X is the \textbf{discrete logarithm} of $y$ with base $a$, modulo $p$.
}

\flashcard{The \blank{discrete logarithm} is the inverse of exponentiation.}

\card{
  Why can a private key in the ElGamal cryptosystem not, in practice, be recovered
  using the public key when $p$ is large?
}{
  To calculate a public key, $y$, a private key, $x$ is needed. The equation for
  modular exponentiation can be used to generate the public key: $y = g^x \bmod{p}$
  where $g$ is a primitive root of the modulus $p$. \\
  It is considered a one-way, or trapdoor function - easy to compute, hard to invert.
  For a large $p$, one of the few ways to figure out the private key x would be to calculate
  $g^x \bmod{p}$ for every $x$ in $1 \leq x < p$ and find when one of these results matches $y$
}

\card{
  What is one way you can argue correctness of Euclid's algorithm?
}{
  Let $r = a mod b$. $hcf(a,b)=hcf(b,r)$ because all factors of a and b are also factors of b and r and vice versa. If they have the same factors, they have the same highest common factor.
}

\card{
  What would half the correctness proof be for Euclid's algorithm?
}{
  As $r=a \bmod{b}$, $\exists q$ such that $a=bq+r$, $\therefore r = a-bq$. \\
  Suppose x is a factor of a and b, then $\exists y and z$ such that $a=xy$, $b=xz$. \\
  Hence: $r=xy-xzq$, $r=x(y-zq)$. \\
  $\therefore x$ is a factor of r (and also of b and r).
}

\flashcard{
  $(a.b)modk=$ \blank{$(amodk.bmodk)modk$}
}

\card{
  Let $p$ be a prime number. What is meant by a primitive root modulo $p$?
}{
  The numbers $r_x$ between 1 and $p-1$ that, when raised by the numbers between 1 and $p-1$ compute all the numbers between 1 and $p-1$ in some order with no repetitions.
}
