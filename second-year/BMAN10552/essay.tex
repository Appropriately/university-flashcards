% There are three recurring questions for the last question, these will cover two of them because I'm lazy
% Discuss what you know about the fundamentals of corporate governance
\card{What is \textbf{Corporate Governance}?}{
    Corporate governance is the system of rules, practices and processes by which a firm is directed and controlled. Corporate governance essentially involves balancing the interests of a company's many stakeholders, such as shareholders, management, customers, suppliers, financiers, government and the community.
}

\card{What are the three categories of business organization?}{
    \begin{flushleft}
        > Sole proprietorship \\
        > Partnership \\ 
        > Corperation 
    \end{flushleft}
    Three categories
}

\flashcard{
    With a Sole proprietorship \& Partnership, the business has a \blank{limited life}. \blank{Transferring ownership} is difficult and \blank{cash for investment} can't be raised.
}

\card{When it comes to taxes, whats the differences between \textit{Sole proprietorship/Partnership} \& \textit{Corperation}?}{
    Taxes are taxed as personal income for \textit{Sole proprietorship/Partnership} and at the corporate rate for a \textit{Corperation}.
}

\flashcard{
    With a Corperation, there is \blank{seperation of ownership and control}. \\ 
    \blank{Shareholders} own firms and represented by board of directors or supervisory board. \\
    Managers run firms and selected by the boards.
}

\card{What do we mean my \textbf{Institutional environments}?}{
    The elaboration of rules and requirements to which individual organizations must conform in order to receive legitimacy and support.
}

\flashcard{
    \begin{flushleft}
        Common Law: \\ 
        \blank{Law is developed through court rulings}. \\
        \blank{Flexible and can adjust quickly to events}. \\ 
        Civil Law: \\ 
        \blank{Law is developed through regularion and code of law}. \\
        \blank{Based on code of principles, does not change}.
    \end{flushleft}
    Civil Law vs. Common Law
}

\card{What are some of the \textit{investor protection mechanisms}?}{
    \begin{flushleft}
        - Proxy vote by mail is allowed. \\ 
        - Votes are not blocked before the annual general meeting. \\ 
        - Cumulative voting or proportional representation exists. \\ 
        - Pre-emptive rights. \\ 
        - Opposed minorities mechanisms. \\ 
        - Minimum percentage to call an extraordinary shareholder's meeting.
    \end{flushleft}
    Investor protection mechanisms
}

% Discuss what you know about the fundamentals behavioral finance
\flashcard{
    Effective market hypothesis (EMH) makes many predictions that does not hold up \blank{empirically}. Stock return are supposed to be unpredictable due to competition to arbitrage. But \blank{empirical} evidence shows many pricing anomalies where apparent mispricings are not exploited away.
}

\card{In the context of behavioral finance, describe \textbf{Overconfidence}.}{
    People in general overestimate their abilities. Investors overestimate the precision of their beliefs and forecasts. This may explain differences in investment performance between gender. E.g, Men are more overconfident \& trade more than woman.
}

\card{In the context of behavioral finance, describe \textbf{Framing}.}{
    An individual may: Reject a bet when it is posed in terms of risk surrounding possible gains or accept that same bet when described in terms of the risk surrounding possible losses. Individuals may act risk averse in terms of gains but risk seeking in terms of loss.
}

\card{In the context of behavioral finance, describe \textbf{Representativeness}.}{
    People commonly do not take into account the size of a sample and believe that small sample is just as representative of a population as large one. Infer pattern too quickly on small sample and extrapolate apparent trends too far into future.
}

% Discuss the concepts of portfolio diversification, systematic risk, and CAPM. Give examples and use diagrams where appropriate. 

