% ========= Lecture 01 =========

\flashcard{
  What do the following stand for? \\
  - $PV$: \blank{Present Value} \\
  - $FV$: \blank{Future Value}
}

\card{
  Given some $PV$, a value for $r$ (year period interest rate) and a number of years $t$, how would you calculate $FV$?
}{
  $FV=PV(1+r)^t$
}

\card{
  How would you calcualte $EAR$ (Effective annual rates)?
}{
  $EAR=(1+\frac{r}{m})^m - 1$ \\
  \begin{flushleft}
    - $r$ is the quoted rate \\
    - $m$ is the number of compounding per year
  \end{flushleft} 
  are the equation's values.
}

% ========= Lecture 02 =========

\flashcard{
  NPV (\blank{Net Present Value}) is the difference between the market value of a project and its cost.
}

\card{
  What is the equation to calculate the NPV?
}{
  $NPV=-IO+\sum\limits_{t=1}^N (\frac{CF_t}{(1+r)^t})$ \\
  \begin{flushleft}
    - $IO$ is the initial investment. \\
    - $N$ \& $t$ are number of years and current year respectively. \\ 
    - $r$ is estimated required return. \\
    - $CF$ is cash flow.
  \end{flushleft}
  Formula variables
}

\card{
  What is Average Accounting Return (AAR) and how do you calculate it?
}{
  $AAR=\frac{Average Net Income}{Average Book Value}$ \\
  The average project earnings after taxes and depreciation, divided by the average book value of the investment during its life. 
}

\flashcard{
  The \blank{Payback Period} is the amount of time required for an investment to generate cash flows sufficient to recover its initial cost.
}

\flashcard{
  If two investments are \blank{Mutually Exclusive}, then taking one of them means that we cannot take the other.
}

% ========= Lecture 3 =========

\card{
  What is a \textbf{bond indenture}?
}{
  The contract between company and the bondholders and includes: \\
  \begin{flushleft}
    - The basic terms of the bonds. \\
    - The total amount of bonds issued. \\
    - A description of property used as security, if applicable. \\
    - Call provisions. \\
    - Details of protective covenants. 
  \end{flushleft}
  Bond Indenture
}

\flashcard{
  \blank{Term structure} is the relationship between time to maturity and yield, all else equal.
}

% ========= Lecture 4 =========

\card{
  What is a dividend?
}{
  A sum of money paid regularly (typically annually) by a company to its shareholders out of its profits (or reserves).
}

\card{
  If you invest in a stock, you can receive cash in two ways. What are those ways?
}{
  \begin{flushleft}
    - The company may pay dividends. \\
    - You may sell your shares to another investor.
  \end{flushleft}
  The ways to receive cash from stock
}