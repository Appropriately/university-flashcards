% ========= Lecture 1 =========

\flashcard{
  What do the following stand for? \\
  - $PV$: \blank{Present Value} \\
  - $FV$: \blank{Future Value}
}

\card{
  Given some $PV$, a value for $r$ (year period interest rate) and a number of years $t$, how would you calculate $FV$?
}{
  $FV=PV(1+r)^t$
}

\card{
  How would you calcualte $EAR$ (Effective annual rates)?
}{
  $EAR=(1+\frac{r}{m})^m - 1$ \\
  \begin{flushleft}
    - $r$ is the quoted rate \\
    - $m$ is the number of compounding per year
  \end{flushleft} 
  are the equation's values.
}

% ========= Lecture 2 =========

\flashcard{
  NPV (\blank{Net Present Value}) is the difference between the market value of a project and its cost.
}

\card{
  What is the equation to calculate the NPV?
}{
  $NPV=-IO+\sum\limits_{t=1}^N (\frac{CF_t}{(1+r)^t})$ \\
  \begin{flushleft}
    - $IO$ is the initial investment. \\
    - $N$ \& $t$ are number of years and current year respectively. \\ 
    - $r$ is estimated required return. \\
    - $CF$ is cash flow.
  \end{flushleft}
  Formula variables
}

\card{
  What is Average Accounting Return (AAR) and how do you calculate it?
}{
  $AAR=\frac{Average Net Income}{Average Book Value}$ \\
  The average project earnings after taxes and depreciation, divided by the average book value of the investment during its life. 
}

\flashcard{
  The \blank{Payback Period} is the amount of time required for an investment to generate cash flows sufficient to recover its initial cost.
}

\flashcard{
  If two investments are \blank{Mutually Exclusive}, then taking one of them means that we cannot take the other.
}

% ========= Lecture 3 =========

\flashcard{A \blank{bond} is a fixed income investment in which an investor loans money to an entity (typically corporate or governmental) which borrows the funds for a defined period of time at a variable or fixed interest rate. }

\card{
  What is the difference between debt and equity?
}{
  Debt is not an ownership interest. Creditors do not have voting rights. Interest is considered a cost of doing business and is tax-deductible. Excess debt can lead to financial and bankruptcy. \\
  Equity: Common stockholders vote to elect the board of directors and on other issues. Dividends are not considered a cost of doing business and are not tax deductible. Stockholders have no legal recourse if no dividends are declared. An all-equity firm cannot go bankrupt.
}

\card{
  What is a \textbf{bond indenture}?
}{
  The contract between company and the bondholders and includes: \\
  \begin{flushleft}
    - The basic terms of the bonds. \\
    - The total amount of bonds issued. \\
    - A description of property used as security, if applicable. \\
    - Call provisions. \\
    - Details of protective covenants. 
  \end{flushleft}
  Bond Indenture
}

\card{
  What do you call bond price due to changes in discount rates?
}{
  Interest rate risk
}

\flashcard{
  \blank{Term structure} is the relationship between time to maturity and yield, all else equal.
}

% ========= Lecture 4 =========

\card{
  What is a dividend?
}{
  A sum of money paid regularly (typically annually) by a company to its shareholders out of its profits (or reserves).
}

\card{
  If you invest in a stock, you can receive cash in two ways. What are those ways?
}{
  \begin{flushleft}
    - The company may pay dividends. \\
    - You may sell your shares to another investor.
  \end{flushleft}
  The ways to receive cash from stock
}

\card{
  What is the \textbf{P/E ratio}?
}{
  Price per share divided by earnings per share. This is most commonly used in real-world for stock valuation. Applied through comparative valuation. If the P/E is high, it is considered overpriced so sell!
}

% ========= Lecture 5 =========

\flashcard{
  The \blank{risk premium} is the return over and above the risk-free rate. The “extra” return earned for taking on risk. Treasuring bills are considered to be risk-free.
}

\card{
  What is \textbf{total cash return} and \textbf{percentage return}?
}{
  Total cash return: Income from investment + Capital gain (loss) due to change in price. \\
  Percentage return: $dividend yield = income / beginning price$ + $capital gains = (ending - starting price) / starting price$.
}

\flashcard{
  \blank{Return variability} is the volatility of asset returns: greater volatility = greater uncertainty or risk. Measured by the variance and standard deviation.
}

\card{
  What is a Security?
}{
  A security is a fungible, negotiable financial instrument that holds some type of monetary value. It represents an ownership position in a publicly-traded corporation (via stock), a creditor relationship with a governmental body or a corporation (represented by owning that entity's bond), or rights to ownership as represented by an option.
}

\card{
  What are the effects of positive and negative news on a security?
}{
  Positive news about a firm increases demand for its security, which in turn pushes up the price of this security. This is reversed for negative news. Demand for security is decreased which pushes down the price. Prices quickly reflect all available public information.
}

\flashcard{To earn higher return, one must \blank{invest in security of firms with higher risk}.}

\card{
  What are the differences between Weak, Semi-strong and Strong form?
}{
  \begin{flushleft}
    - Weak form: Prices reflect all past market information such as price and volume. If the market is weak form efficient, then investors cannot earn abnormal returns by trading on market information. \\
    - Semi-strong form: Prices reflect all publicly available information including trading information, annual reports, press releases, etc. \\
    - Strong form: Prices reflect all information: including public and private.
  \end{flushleft}
  Difference between the three forms
}

% ========= Lecture 6 =========

\flashcard{\blank{Expected returns} are based on the probabilities of possible outcomes. In this context, “expected” means “average” if the process is repeated many times.}

\flashcard{\blank{Variance and standard deviation} still measure the volatility of returns. Using unequal probabilities for the entire range of possibilities. Weighted average of squared deviations.}

\card{
  What is an \textbf{Asset}?
}{
  An asset is a resource with economic value that an individual, corporation or country owns or controls with the expectation that it will provide future benefit. Assets are reported on a company's balance sheet, and they are bought or created to increase the value of a firm or benefit the firm's operations. 
}

\card{
  What is a \textbf{Portfolio}?
}{
  A portfolio is a collection of assets. An asset’s risk and return are important to how the stock affects the risk and return of the portfolio.
}

\flashcard{\blank{Portfolio diversification} is the investment in several different asset classes or sectors.}

\card{
  What are the differences between \textbf{Systematic \& Unsystematic risk}?
}{
  \begin{flushleft}
    Systematic risk is the risk factors that affect a large number of assets. Includes such things as changes in GDP, inflation, interest rates, etc. \\
    Unsystematic risks are risk factors that affect a limited number of assets. Also known as unique risk and asset-specific risk. Includes such things as labour strikes, part shortages, etc.
  \end{flushleft}
  Systematic VS. Unsystematic risk
}

% ========= Lecture 7 =========