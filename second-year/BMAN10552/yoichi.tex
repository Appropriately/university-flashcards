% ========= Lecture 0 =========

\card{
  What is the \textbf{Time Value of Money}?
}{
  The time value of money (TVM) is the idea that money available at the present time is worth more than the same amount in the future due to its potential earning capacity. This core principle of finance holds that, provided money can earn interest, any amount of money is worth more the sooner it is received.
}


% ========= Lecture 1 =========

\flashcard{
  \blank{Compound interest} (or \blank{compounding interest}) is interest calculated on the initial principal and also on the accumulated interest of previous periods of a deposit or loan. 
}

\card{
  What is an \textbf{Annuity}?
}{
  An annuity is a financial product that pays out a fixed stream of payments to an individual, primarily used as an income stream for retirees.
}

\flashcard{
  \blank{Perpetuity} refers to an infinite amount of time. In finance, it is a constant stream of identical cash flows with no end.
}

\card{
  How would you calcualte $EAR$ (Effective annual rates)?
}{
  $EAR=(1+\frac{r}{m})^m - 1$ \\
  \begin{flushleft}
    - $r$ is the quoted rate \\
    - $m$ is the number of compounding per year
  \end{flushleft} 
  are the equation's values.
}

% ========= Lecture 2 =========

\flashcard{
  NPV (\blank{Net Present Value}) is the difference between the market value of a project and its cost.
}

\card{
  What is the equation to calculate the NPV?
}{
  $NPV=-IO+\sum\limits_{t=1}^N (\frac{CF_t}{(1+r)^t})$ \\
  \begin{flushleft}
    - $IO$ is the initial investment. \\
    - $N$ \& $t$ are number of years and current year respectively. \\ 
    - $r$ is estimated required return. \\
    - $CF$ is cash flow.
  \end{flushleft}
  Formula variables
}

\card{
  What is Average Accounting Return (AAR) and how do you calculate it?
}{
  $AAR=\frac{Average Net Income}{Average Book Value}$ \\
  The average project earnings after taxes and depreciation, divided by the average book value of the investment during its life. 
}

\flashcard{
  The \blank{Payback Period} is the amount of time required for an investment to generate cash flows sufficient to recover its initial cost.
}

\flashcard{
  \blank{Liquidity} describes the degree to which an asset or security can be quickly bought or sold in the market without affecting the asset's price.
}

\card{
  What are the advantages \& disadvantages of the payback period calculation?
}{
  \begin{flushleft}
    + Easy to understand \\
    + Biased towards liquidity \\
    + Adjusts for uncertainty of later cash flows \\
    - Ignores the time value of money \\
    - Requires an arbitrary cutoff point \\
    - Ignores cash flows beyond the cutoff date  \\
    - Biased against long-term projects, such as research and development, and new projects
  \end{flushleft}
  Payback period +/-
}

\card{
  What is \textbf{Net present value}? How would you calculate it?
}{
  The difference between the present value of cash inflows and the present value of cash outflows over a period of time. NPV is used in capital budgeting to analyse the profitability of a projected investment or project. \\
  It is calculates by subtracting the initial investment from the present value of cash flows.
}

\card{
  What is the \textbf{Internal rate of return}?
}{
  IRR is the return that makes $NPV = 0$. It is the most important alternative to NPV. You'd accept a project if the IRR is greater than the required return.
}

\flashcard{
  If two investments are \blank{Mutually Exclusive}, then taking one of them means that we cannot take the other.
}

% ========= Lecture 3 =========

\flashcard{A \blank{bond} is a fixed income investment in which an investor loans money to an entity (typically corporate or governmental) which borrows the funds for a defined period of time at a variable or fixed interest rate. }

\flashcard{\blank{Bond yield} is the amount of return an investor realizes on a bond.}

\card{
  What is the difference between debt and equity?
}{
  Debt is not an ownership interest. Creditors do not have voting rights. Interest is considered a cost of doing business and is tax-deductible. Excess debt can lead to financial and bankruptcy. \\
  Equity: Common stockholders vote to elect the board of directors and on other issues. Dividends are not considered a cost of doing business and are not tax deductible. Stockholders have no legal recourse if no dividends are declared. An all-equity firm cannot go bankrupt.
}

\card{
  What is a \textbf{bond indenture}?
}{
  The contract between company and the bondholders and includes: \\
  \begin{flushleft}
    - The basic terms of the bonds. \\
    - The total amount of bonds issued. \\
    - A description of property used as security, if applicable. \\
    - Call provisions. \\
    - Details of protective covenants. 
  \end{flushleft}
  Bond Indenture
}

\card{
  What do you call bond price due to changes in discount rates?
}{
  Interest rate risk
}

\flashcard{
  \blank{Term structure} is the relationship between time to maturity and yield, all else equal.
}

% ========= Lecture 4 =========

\card{
  What is a dividend?
}{
  A sum of money paid regularly (typically annually) by a company to its shareholders out of its profits (or reserves).
}

\card{
  If you invest in a stock, you can receive cash in two ways. What are those ways?
}{
  \begin{flushleft}
    - The company may pay dividends. \\
    - You may sell your shares to another investor.
  \end{flushleft}
  The ways to receive cash from stock
}

\card{
  What is the \textbf{P/E ratio}?
}{
  Price per share divided by earnings per share. This is most commonly used in real-world for stock valuation. Applied through comparative valuation. If the P/E is high, it is considered overpriced so sell!
}

% ========= Lecture 5 =========

\flashcard{
  The \blank{risk premium} is the return over and above the risk-free rate. The “extra” return earned for taking on risk. Treasuring bills are considered to be risk-free.
}

\card{
  What is \textbf{total cash return} and \textbf{percentage return}?
}{
  Total cash return: Income from investment + Capital gain (loss) due to change in price. \\
  Percentage return: $dividend yield = income / beginning price$ + $capital gains = (ending - starting price) / starting price$.
}

\flashcard{
  \blank{Return variability} is the volatility of asset returns: greater volatility = greater uncertainty or risk. Measured by the variance and standard deviation. 
}

\card{What is the effect of a higher and lower \textbf{volatility}?}{
  A higher volatility means that a security's value can potentially be spread out over a larger range of values. This means that the price of the security can change dramatically over a short time period in either direction. A lower volatility means that a security's value does not fluctuate dramatically, and tends to be more steady.
}

\card{
  What is a Security?
}{
  A security is a fungible, negotiable financial instrument that holds some type of monetary value. It represents an ownership position in a publicly-traded corporation (via stock), a creditor relationship with a governmental body or a corporation (represented by owning that entity's bond), or rights to ownership as represented by an option.
}

\card{
  What are the effects of positive and negative news on a security?
}{
  Positive news about a firm increases demand for its security, which in turn pushes up the price of this security. This is reversed for negative news. Demand for security is decreased which pushes down the price. Prices quickly reflect all available public information.
}

\flashcard{To earn higher return, one must \blank{invest in security of firms with higher risk}.}

\card{
  What are the differences between Weak, Semi-strong and Strong form?
}{
  \begin{flushleft}
    - Weak form: Prices reflect all past market information such as price and volume. If the market is weak form efficient, then investors cannot earn abnormal returns by trading on market information. \\
    - Semi-strong form: Prices reflect all publicly available information including trading information, annual reports, press releases, etc. \\
    - Strong form: Prices reflect all information: including public and private.
  \end{flushleft}
  Difference between the three forms
}

% ========= Lecture 6 =========                            

\flashcard{\blank{Expected returns} are based on the probabilities of possible outcomes. In this context, “expected” means “average” if the process is repeated many times.}

\flashcard{\blank{Variance and standard deviation} still measure the volatility of returns. Using unequal probabilities for the entire range of possibilities. Weighted average of squared deviations.}

\card{
  What is an \textbf{Asset}?
}{
  An asset is a resource with economic value that an individual, corporation or country owns or controls with the expectation that it will provide future benefit. Assets are reported on a company's balance sheet, and they are bought or created to increase the value of a firm or benefit the firm's operations. 
}

\card{
  What is a \textbf{Portfolio}?
}{
  A portfolio is a collection of assets. An asset’s risk and return are important to how the stock affects the risk and return of the portfolio.
}

\flashcard{\blank{Portfolio diversification} is the investment in several different asset classes or sectors.}

\card{
  What are the differences between \textbf{Systematic \& Unsystematic risk}?
}{
  \begin{flushleft}
    Systematic risk is the risk factors that affect a large number of assets. Includes such things as changes in GDP, inflation, interest rates, etc. \\
    Unsystematic risks are risk factors that affect a limited number of assets. Also known as unique risk and asset-specific risk. Includes such things as labour strikes, part shortages, etc.
  \end{flushleft}
  Systematic VS. Unsystematic risk
}

% ========= Lecture 7 =========

\flashcard{
  Our cost of capital \blank{provides us with an indication of how the market views the risk of our assets}.
}

\card{What is the \textbf{Beta}?}{
  Beta is a measure of the volatility, or systematic risk, of a security or a portfolio in comparison to the market as a whole. \\ 
  A beta of 1 indicates that the security's price moves with the market. A beta of less than 1 means that the security is theoretically less volatile than the market. A beta of greater than 1 indicates that the security's price is theoretically more volatile than the market.
}

\card{In the \textbf{Capital Asset Pricing Model}, what does the beta tell us?}
{
  $beta < 1$ implies the asset has less systematic risk than the overall market. \\ 
  $beta > 1$ implies the asset has more systematic risk than the overall market. \\
  $beta = 1$ implies that the asset has the same systematic risk as the  overall market.
}

\card{What is the \textbf{Security Market Line}?}{
  The SML is a line drawn on a chart that serves as a graphical representation of the capital asset pricing model (CAPM), which shows different levels of systematic, or market, risk of various marketable securities plotted against the expected return of the entire market at a given point in time. 
}

% ========= Lecture 8 =========

\flashcard{
  The \blank{Cost of Equity} is the return required by the equity investors given the risk of the cash flows from the firm.
}

\card{What are the pros and cons of the \textbf{Dividend Growth Model}}{
  It is easy to understand and use. \\ 
  Only applicable to companies currently paying dividends. Dividends must be growing at a reasonably constant rate. Extremely sensitive to the estimated growth rate. Does not explicitly consider risk.
}

\card{What are the pros and cons of the \textbf{Capital Asset Pricing Model} approach?}{
  Explicitly adjusts for systematic risk & applicable to all companies (as long as we can compute the beta). \\
  You have to estimate the expected market risk premium and beta, both of which vary over time. We are always relying on past data to predict the future.
} 

\flashcard{
  The \blank{Cost of Debt} refers to the effective rate a company pays on its current debt. 
}

\flashcard{
  \blank{Weighted average cost of capital (WACC)} is a calculation of a firm's cost of capital in which each category of capital is proportionately weighted. \\
  The “average” is the \blank{required return on the firm’s assets}, based on the market’s perception of the risk of those assets. \\
  The weights are determined by \blank{how much of each type of financing is used}.
}

% ========= Lecture 9 =========

\flashcard{
  \textbf{Capital Structure} decision involves the choice between \blank{debt} \& \blank{equity} to finance a firm.
}

\card{
  How does leverage affect the \textbf{Earnings Per Share} and \textbf{Return On Equity} of a firm?
}{
  It amplifies the variation. When we increase the amount of debt financing, we increase the fixed interest expense. If we have a really good year, then we pay our fixed costs, and have more left over for our stockholders. If we have a really bad year, we still have to pay our fixed costs, and have less left over for our stockholders.
}

\flashcard{
  Types of bankruptcy costs: \\
  \blank{Direct costs} - directly associated with bankruptcy, e.g. legal and administrative expense. \\
  \blank{Indirect costs} - the cost of avoiding bankruptcy filing by a financial distressed firm.
}

% ========= Lecture 10 =========

\flashcard{
  A \blank{derivative} is a financial security with a value that is reliant upon or derived from an underlying asset or group of assets. The \blank{derivative} itself is a contract between two or more parties based upon the asset or assets.
}

\flashcard{
  A \blank{hedge} is an investment to reduce the risk of adverse price movements in an asset. Normally, it consists of taking an offsetting position in a related security, such as a futures contract.
}

\card{What are \textbf{Futures/Forwards}?}{
  They are contracts in which one party agrees to deliver a specified asset to another party at a specific price and time.
}

\card{
  You buy a call option on a company's stock for \$4 per share, which gives you the option to buy the stock at a strike price of \$60 per share on the experiation date. The time comes, and the stock price is \$65. Should you buy? At which price should you not buy.
}{
  You should, it is a \$1 gain. \\
  Anything less than \$64 is a loss, due to the initial call option price.
}

% ========= Lecture 11 =========

