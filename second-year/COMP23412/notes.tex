% ====== Web Frameworks/ MVC Architectural Design Pattern ======

\flashcard{
  \blank{MVC architecture} is a basic pattern where you separate the model (data), view (display), and controller (logic) into different files and directories.
}

\flashcard{
  \blank{Models} represent knowledge. Could be a single object or a structure of objects.
}

\flashcard{
  A \blank{view} is a (visual) representation of its model
}

\flashcard{
  The \blank{controller} is the link between the user and the system.
}

\flashcard{
  With the MVC Design Pattern, \blank{Model} objects encapsulate the data, doesn't communicate directly with View \& defines the logic, manipulation and processing of the data.
}

\flashcard{
  With the MVC Design Pattern, \blank{View} objects present the data and enables user interaction with it. It communicates with controller, notified about changes in data \& controller notifies of any user-made changes.
}

\card{
  What are the benefits of the MVC architecture?
}{
  \begin{flushleft}
    - More reusable: use the same View for multiple application \\ 
    - Easily adaptable: Each object has a clearly defined role. Good design principle
  \end{flushleft}
  Benefits fo the MVC architecture
}

% ====== User Interface Design ======

\card{
  What are the stages of UX design?
}{
  $Sketches \longrightarrow Wireframe \longrightarrow Mockups \longrightarrow Prototypes$
}

\card{
  Why would you produce Mock-ups in the first place?
}{
  \begin{flushleft}
    - Dialogue with customers; can be used to confirm requirements, show different choices \& exchange ideas. \\ 
    - Acts as a form of testing, by preventing misunderstanding and removing bugs early.
  \end{flushleft}
  Reasons for Mockups
}

\card{
  What are the 8 golden rules of user interface design?
}{
  \begin{flushleft}
    Strive for consistency \\ 
    Seek universal usability \\
    Offer informative feedback \\
    Design dialogues that bring closure \\
    Prevent errors \\
    Permit easy reversal \\ 
    Keep Users in control \\
    Reduce short term memory load
  \end{flushleft}
  8 golden rules
}

% ====== Modelling Data ======

\flashcard{
  In Spring “model” and “\blank{entity}” used interchangeably.
}

\card{
  In the context of Spring, what is a \textbf{Repository}?
}{
  Data lives in a repository. Repositories are the Spring Mechanism for querying the underlying DB. We used CRUD (Create, Read, Update, Delete) repository.
}

% ====== Testing Functionality in Isolation ======

\card{
  What are some important features of unit testing?
}{
  Test’s don’t build on other tests. They test one thing. Stay within class/process/network boundaries - don’t test database as a side effect.
}

\card{
  Integration Testing: Testing the system from end to end. What are the steps?
}{
  Client submits a request to the web server. Web server maps request to a controller. Controller gets data via the DAO layer which gets data from DB. Controller passes data to view. View is processed. Web server sends view to client.
}

\flashcard{
  \blank{Dummy} objects are passed around but never actually used. Usually they are just used to fill parameter lists.
}

\flashcard{
  \blank{Fake} objects actually have working implementations, but usually take some shortcut which makes them not suitable for production.
}

\flashcard{
  \blank{Stubs} provide canned answers to calls made during the test, usually not responding at all to anything outside what's programmed in for the test.
}

\flashcard{
  \blank{Mocks} are pre-programmed with expectations which form a specification of the calls they are expected to receive. They can throw an exception if they receive a call they don't expect and are checked during verification to ensure they got all the calls they were expecting.
}

% ====== Integrating external Services ======

\card{
  What is Spring Social?
}{
  Framework of Spring Boot; establishes connections between Spring boot apps and SaaS(Software as a service) providers e.g. Twitter, Facebook. \\
  $SaaS = API + resources + Interface$
}

% ====== Meeting Customer Requirements ======

\flashcard{
  An \blank{acceptance test} is a formal description of the behavior of a software product, expressed as a example or a usage scenario.
}

\card{
  What are the benefits of \textbf{Acceptance Tests}?
}{
  \begin{flushleft}
    Closer collaboration between developers and user/customer \\
    Providing clear and unambiguous “contract” \\
    Decrease chance and severity of defects 
  \end{flushleft}
  Acceptance tests benefits
}

% ====== RESTful WebServices and Test Driven ======

\flashcard{
  \blank{Representational State Transfer (REST)} is an architectural style that is the underlying architectural principal of the WWW. Clients can operate without knowing anything about \blank{the server} \& \blank{the server's resources}. Client and server must agree on the \blank{media type} used.
}

\card{
  What does \textbf{RED}, \textbf{GREEN} \& \textbf{REFACTOR} mean in the context of \textbf{Test Driven Development (TDD)}? 
}{
  \begin{flushleft}
    - Write a test that does not work (RED) \\
    - Make the test work (GREEN) \\
    - Improve the code and eliminate duplication (REFACTOR)
  \end{flushleft}
  \textbf{Test Driven Development (TDD)}
}

\card{
  Give some reasons why you would use \textbf{Test Driven Development}?
}{
  \begin{flushleft}
    - Quality Assurance becomes proactive rather than reactive. \\ 
    - Estimations can be accurate enough to involve real customers in daily development. \\ 
    - Short iterations. Each iteration produces a working product. \\ 
    - Encourages good OO design practise. \\ 
    - Encourages design for testability. \\ 
    - Get an unambiguous progress meter. \\ 
    - Build up a set of regression test as we go along.
  \end{flushleft}
  Test Driven Development
}