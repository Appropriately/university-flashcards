\card{
  What are the three main categories of permanent storage media?
}{
  Write Once Read Many, Write Many Read Many, Write (not too) Many Read Many
}

\card{
  What are the three terms used to quantify hard drive performance
  characteristics?
}{
  Seek time, search time and transfer rate.
}

\card{
  What is the seek time?
}{
  The time it takes for the head to reach the target track on the platter.
}

\card{ 
  What is the search time?
}{
  The time it takes for the target sector to arrive under the head.
}

\card{
  What is the transfer rate?
}{
  The amount of data that can be read/written per unit time.
}

\card{
  Give the equation for disk access time.
}{
  \[
    \text{Disk access time} = \text{Seek time} + \text{Search time}
                                + \text{Transfer time}
  \]
}

\card{
  How can we work out the search time from the RPM?
}{
  \[
    \text{Search time (seconds)} = \frac{0.5\text{ rotations} \times 60}{RPM}
  \]
}

\card{
  What happens if the OS wants to read a file that is split over multiple
  sectors on the hard drive?
}{
  The processor in the hard drive changes the read order to the most efficient
  order of sectors for the head to minimise wasted rotations.
}

\card{
  What does RAID stand for?
}{
  Redundant Array of Independent Disks
}

\flashcard{
  RAID0 has a \blank{high} transfer rate and a \blank{low} redundancy because
  data is \blank{striped} over multiple disks.
}

\flashcard{
  RAID1 has a \blank{low} transfer rate and a \blank{high} redundancy since data
  is \blank{mirrored} over multiple disks.
}

\flashcard{
  We can use \blank{error correction} such as \blank{hamming codes} and
  \blank{parity bits} to provide redundancy using RAID.
}

\card{
  Define RAID{2,3,4,5,6}.
}{
  \begin{tabularx}{0.4\textwidth}{lX}
    2 & Bit striping and hamming codes\\
    3 & Byte striping and parity bits\\
    4 & Block striping and parity\\
    5 & Block striping and distributed parity\\
    6 & Double distributed parity
  \end{tabularx}
}

\card{
  What is the transistor used in SSD's?
}{
  A Floating Gate Field Effect Transistor.
}

\card{
  What is wear-levelling?
}{
  When a SSD maps different logical addresses to different physical blocks so
  that all blocks are worn out at the same rate.
}

\flashcard{
  RAID lets us break the concept of \blank{file systems} by mapping them
  onto \blank{multiple drives} using \blank{striping and mirroring}. The actual
  details of storage are \blank{abstracted away}.
}

\card{
  What is a SAN?
}{
  A Storage Area Network.
}

\flashcard{
  \blank{ZFS} is a volume aware filesystem. It protects against losing files,
  running out of space, corruption of data etc by being very flexible and
  having lots of \blank{ECC} and implementing
  \blank{copy}-\blank{on}-\blank{write}, simple \blank{rollback and recovery},
  \blank{wear leveling}, self \blank{checking} and \blank{healing}, sumchecking
  and more.
}